\section{24.09.04}

Created my own Gitlab runner. Out of an old laptop. It did not work out of the box, but I hope
I can get it work by the end of today.

(1 minute later)

I dont know why, but my runner is disabled for 1 hour. Cool beans.

\dots

I re-installed debian on the old laptop, and registered the runner on the laptop, instead of
through SSH, but this did not work either. I also added a runner on my current laptop, and
this worked, somehow. But it was a flakey process. I saw on some forums that I should run
gitlab-runner with sudo, but this did not work on my machine, neither did without, it was
just stuck, it seemed. I finally ran it with the --debug flag, and then it worked.

I can now see that downloading 2G might be slower than just building it.

The pipelines take a ridicolus amount of time to run, so I should look into splitting them up
there is no need to run them for every little commit. This is a tomorrow me's problem, however.
I am also noticing that my desktop is lagging when being used as a runner, so I should enable
the shared runners for those jobs that it can run (by using tags).

I also need to check out the test coverage being created for nmide-core, it is not working for
some reason.


\section{24.09.05}

I forgot to turn on my gitlab-runner before going to JAFU, so now I can't build unless I create
a new gitlab-runner. I also can't ssh into my old laptop, because I don't know the ip, nor if
that port is open on our network.

So no ci work today.

I know of other problems I can/should work on, like the Rust ABI. It is not stable. Which is a
major problem. The Rust Compiler does a lot of optimalizations on data-types, which means across
different semver's, the same struct can have different orientation in memory, which leads to
undefined behavior. There are crates that can fix this, like stabby, but this will severely change
the nmide-std-lib API. But this will make Plugin development a real pain, which means I would have
to include more tests for memory leakage, functions for html-element creation/manipulation and
possibly macros.

So, I have an idea! I'll keep the same general structure on the "old" Html stuff, but use
ManuallyDrop wrapper on the fields that are used across the ABI, kids, attrs, and text. That way I
can ensure that the value that I read exists, and basically run a garbage collector every now and
then.

I need to do a deep-dive into crates like stabby and safer_ffi, to make my Rust ABI development life
easier.


\section{24.09.06}

I need a CLI-tool for creating plugins. Because I need a lot of boilerplate code for it too work,
safely and stably, in the Nmide environment. If I do not do this, I risk of creating Magnolia IDE
2, Electric Boogaloo. Which, now that I think about it, I risk either way. If I don't use crates to
make the Rust ABI stable across compiler versions, the application is kinda future safe, because for
every major/minor compiler update, you only need to recompile the IDE and Plugins you are using, and
any future compiler update will likely not break the ABI, or make it anymore unstable on the same
compiler version. While, on the other hand if I rely on crates ensuring ABI safety, a future Rust
version could break it, (insert the 1.82.0 to 1.81.0 minor update that broke a lot of crates). I
think with the crate way, I or other people could eventually fix that which is broken, but I can't
guarantee it. Working with ABI/FFI's is a pain.

I think I should work on something else for a while.
