\section{Background}

Traditional IDEs encompass essential features such as syntax highlighting, code navigation, and hover-help,
playing a crucial role in the software development process. However, their limitations become apparent when
working with experimental languages. The paper advocates for modularization and composability as key design principles,
demonstrating their ability to extend the operational lifespan of software by adapting to new paradigms and tools.
The discussion revolves around Magnolia, a generic programming language developed at UIB, as a case study
illustrating the need for a specialized IDE. It is a way to experiment with generic programming.
And too achieve this in a sufficient manner, a more specialized IDE is required.

%% The point of this sub-section, is to explain what happened previously at UIB, and
%% why this is the best way to do it
\subsection{Background at UIB}

%% Some paragraphs about what happened before
The previous IDE for Magnolia, was an (old?) version of Eclipse, using some plugins/features that was
outdated (around some date). This IDE's life time was limited by external plugins/features that where 
not maintained by UIB. This meant that for future development of Magnolia, an outdated IDE was needed, with
outdated software and (other reasons this was not good). A solution could be to use something like
"Visual Studio Code", which is a very popular IDE(source?), more popular than Eclipse (source?).

%% Why not VS-Code?
%% TODO: Add some mentions of open-source
Then new development of the new IDE, then, would be turned around, instead of maintaining an IDE, the plugins
for the IDE would be maintained. While it is unlikely that "Visual Studio Code" will be deprecated in the
near future, it could still happen. The best solution is to keep everything internal.

%% Probably not academic enough
But creating any IDE would still limit the lifetime of the application. The best example of a long living active
IDE, or, atleast editor, is Vim(source?). Vim is not a feature full editor, but it is simple, lightweight, and works.
But most people use it, for how easy it is to extend; It's lifetime has been greatly extended by the ease of modularization.
Any popular plugin for Vim is open-source, and therefore, if any plugin had an active community around it,
if the \textit{lead} developer of the plugin stopped developing it, that community continue developing the plugin, either
by getting maintenance access to the repository, or by forking it. Ensuring the lifetime of the plugin.

%% Red-thread is missing, find it
%% Should probably also expand on what we mean by limiting (i.e. limiting in the sense of plugins being developed)
\paragraph{Language Agnostic} The largest limiting factor in plugin oriented applications, is the \textit{language barrier}
Most applications limit what language one can extend an application with, like in "Visual Studio Code",
where it's \textit{just} JavaScript/Html/Css. Or Intellij, where one can use Java or Kotlin. (Which doesn't count as
two different languages, it's just two, because both Java and Kotlin both compile to JVM). But what does language agnostic
mean in the context of programming languages? It is, and always will be (sorry Rust): C. Any language worth a damn, can
create some bindings with C. So, for an plugin application to be Language Agnostic, it must be able to use C Plugins.
This just means that the application must be able to call function from a C library.

So, to get the best application (lifetime-vise), it should fulfill all these criteria:

\begin{enumerate}
  \item Modular
  \item Open Sourced
    \begin{enumerate}
      \item Both for Plugins
      \item and for the Application
    \end{enumerate}
  \item Language Agnostic
\end{enumerate}


