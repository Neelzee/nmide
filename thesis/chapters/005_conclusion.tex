\chapter{Conclusion \& Discussion} \label{cha:conclusion}

In this chapter, we will discuss some of the issues encountered when developing
this modular \gls*{ide}. In section \ref{sec:mod-dev}, we will discuss our
experience developing modules for this \gls*{ide}, and how that was affected by
the development of the architecture itself. We will continue in a similar vein,
by discussing some of the issues with the \gls*{api}, in section
\ref{sec:lackluster}, and how they were \textit{solved}. Section \ref{sec:lla}
will discuss language agnosticism, while section \ref{sec:fm} will highlight the
challenges for development of foreign modules. And finally, we will make a
conclusion from our hypothesis \ref{hyp:modular}.


\section{Modular development} \label{sec:mod-dev}

In this thesis, we have shown that developing against a zero-core modular
architecture is trivial. By utilizing separation of concerns, a module developer
needs to only understand the feature they want to extend, or if it is an
entirely new feature, find out what has been done before.


\subsection{Unstable API}

Developing against an unstable \gls*{api} is difficult, and when developing a
module architecture, it is like an unstable \gls*{api} when it is not
\textit{mature}, e.g. when it does not have settled modules to develop against.
Since this is the case, there are a lot of issues with the existing modules,
making the user experience less than competing \gls*{ide}s. Most of these are
minors, and can be fixed with some minor revisions to the existing modules, for
instance, when closing the \gls*{ide}, unsaved changes are discarded, with no
information given to the user. Or how what project a user was working on, is not
saved between instances, so a user has to re-open the project they worked on.
This is a side effect of the development plan, and not the architecture. To
fully test out this architecture, it was thought that a wide range of modules
should be implemented, to quickly iron out issues with the implementation of the
architecture, and to figure out what functionality Tauri has, that we can
expose, like the file selection.

So not only were modules needed to cover the necessities to qualify as an
\gls*{ide}, but they were also needed to \textit{test} the implementation. Not
having a developer dedicated to only implement modules, meant that module
development was usually dropped for other things. As every time a module where
worked on, it would eventually lead to a discovery, that the current \gls*{api}
needed some change, which would enable the module feature to be easier
implemented. A concrete example of this, is the editor module.
An essential part of an editor, in an \gls*{ide} at least, is being able to
utilize a \gls*{lsp}. Most of the communication between a client and \gls*{ls},
require information about \textit{where} the user is in the text. This
information is available in a \textit{textarea}-element, but some change to how
Events are sent were needed. In standard JavaScript development,
\textit{eventListeners} can be specific to the \gls*{html}-element they are
applied to. The same is not possible in our \gls*{api}, as Events are generic.


\subsection{Module deprecation}

Instead, we made Events gather information about the \gls*{dom}-Event they were
triggered by, so in the case of \textit{click} attribute, we know the
\gls*{dom}-event is of type \textit{MouseEvent}, which can give us some,
information. And if the \textit{target}, (a field on \textit{MouseEvent}) is an
instance of \textit{HtmlInputElement} or \textit{textarea}, we know that the
\textit{selectionStart} and \textit{value} field exist on the target. With
which, we can manually calculate the position of the click. Implementing this
meant adding a breaking change to the \gls*{api}, which deprecated different
modules, so more time was spent on re-implementing them.


\section{Inconsistent UI and ad-hoc solutions for lackluster API} \label{sec:lackluster}

There is an issue in the current \gls*{api} on maintaining consistency between
the \gls*{ui} in the frontend, and the \gls*{ui} representation the \gls*{rsms}
has access to. This means that there is no good way to achieve saving of an
edited file. But this is still a feature the editor module supports. The way
saving was implemented, was to add plain JavaScript to the save button, where
when the user presses it, the contents of the textarea are sent to the ide\_fsa
module, which can save the contents. This is an action that bypasses the core
\gls*{ide}, which was necessary due to the lacking \gls*{api}.


\section{Lacking language agnosticism} \label{sec:lla}

Not really achieved, because we cannot syntactically translate between a
JavaScript Module and a Rust Module. This is due to the differences between the
utility libraries created for \gls*{jsms} and \gls*{rsms}. When \gls*{jsms} modules
where created, they were primarily made for using existing JavaScript libraries,
to showcase this interoperability. So, much of the \gls*{html} elements were
created using JavaScript, so the utility library primarily focused on this,
having builder pattern for creating \gls*{html}. There is a similar builder
pattern in the Rust utility library, but it is not a one-to-one mapping, meaning
there are some semantic differences between two modules doing the same.

But \textit{installation} of the modules also differ. In JavaScript, module
developers can simply invoke the \textit{installModule} function, with their
created module, to install the module. The reason this works, is that when we
bundle all JavaScript modules during compile time, it ends up as a script-tag in
the \gls*{dom}. The same is for the case of runtime JavaScript modules. The
result, in either case, is that the entire contents of the JavaScript file is
evaluated, meaning even though we are simply importing a JavaScript file, and
not explicitly invoking anything, it ends up with the modules being installed.

This is not the case in Rust, importing another Rust crate does not mean we
invoke it. That is why we need the extra steps of creating a
\textit{ModuleBuilder}, which has to implement the \textit{ModuleBuilder} trait,
so that we can build the module.


\section{Foreign modules} \label{sec:fm}

Languages like Gleam and PureScript, which compile directly to JavaScript can
be trivially added. But for languages that can target the C-\gls*{abi}, this is
less trivial. This is because of how the core-\gls*{ide} was designed. We decided
to use a \textit{Rust-y} approach, meaning we utilized many of the features that
made interoperability between the Rust-\gls*{abi} and C-\gls*{abi} more complex.
An example of this, can be found in the listing \ref{lst:value} and
\ref{lst:rsValue}, where we have the \textit{standard} value variant, and then
the \textit{C-safe} variant.

\begin{code}[H]
  \lstinputlisting
    [ language=Rust
    , caption={Value variant (Rust)}
    , label=lst:value
    , firstline=19
    , lastline=31
    ]{./libs/rust/core-std-lib/src/state/mod.rs}
\end{code}

\begin{code}[H]
  \lstinputlisting
    [ language=Rust
    , caption={C-safe value variant (Rust)}
    , label=lst:rsValue
    , firstline=16
    , lastline=21
    ]{./libs/rust/foreign-std-lib/src/state/rs\_state.rs}
\end{code}

Note the \textbf{\#[repr(C)]} macro attribute, and the two fields,
\textit{kind} and \textit{val}. The macro attribute specifies to the Rust
compiler that it should \textit{do what C does}. This is in regard to order,
size and alignment of fields of a structure. Since we cannot have the same enum
structure as we can in Rust, the work-around was an enum that specifies what
kind of value we are working with (\textit{val}), and a union, that holds the
specific value. A union in both C and Rust, has the same size in memory, as the
largest possible value it can store. In listing \ref{lst:rsValueUnion} we can
see this union. Accessing a field is inherently an \textit{unsafe} action, as we
cannot tell the compiler if the bytes we are reading are actually and integer,
or is a list of values. We can see this, as in the listing
\ref{lst:rsValueUnsafe}, on line three, we have to use the \textit{unsafe}
keyword in Rust, which essentially means the compiler cannot promise what we are
doing in this code block is \textit{valid}.

\begin{code}
  \lstinputlisting
    [ language=Rust
    , caption={Union used to hold the values the C-safe value can have (Rust)}
    , label=lst:rsValueUnion
    , firstline=134
    , lastline=144
    ]{./libs/rust/foreign-std-lib/src/state/rs\_state.rs}
\end{code}

\begin{code}
  \lstinputlisting
    [ language=Rust
    , caption={
      Accessing a value in the C-safe value variant is inherently unsafe (Rust)
    }
    , label=lst:rsValueUnsafe
    , firstline=53
    , lastline=59
    , numbers=left
    , numberstyle=\tiny\color{gray}
    ]{./libs/rust/foreign-std-lib/src/state/rs\_state.rs}
\end{code}

But with the starting point of the runtime Rust module system, a C module system
could be developed. One would just have to ensure that the differences between
the modules are syntactical, and not semantics.


\section{Conclusion} \label{sec:conclusion}

Developing against an unstable \gls*{api}, means that modules can be
deprecated. It also means that module language agnosticism can quickly
disappear, since that depends on having multiple different libraries in sync
with an unstable one. In fact, many of the issues that we have claimed to be
innate with \gls*{ide}s, appear in this stage of our modular architecture. But,
our \gls*{api} is bounded, we have some types, and some operation on those
types. We have chosen to have a larger set of operations, simply due to the
fact that this enhances the module developer experience. But to figure out what
utility functions are necessary, we need to develop modules. Once the
satisfactory functions are developed, our \gls*{api} is stable. Which means
modules are no longer in danger to be deprecated. Which means that module
language agnosticism can be corrected for. Finally, this means that future
changes coming from the outside, being a paradigm shift on what is necessary to
have in an \gls*{ide}, or a language, the necessary modules can quickly be
developed and integrated into the current solution.

