\chapter{Introduction}

Standard \gls*{ide}s are indispensable tools in modern software development,
offering features like early bug reporting, project outline visualization, code
highlighting, and code completion, however, these \gls*{ide}s may not adequately
support the unique demands of experimental programming languages. Experimental
languages could introduce new concepts like \gls*{asr} transformations, term
algebras, mathematics of arrays, \gls*{stf}~\cite{stf:haveraaen:2020}, or other novel
programming features. These are concepts from the academic community, and are
not common in \textit{mainstream} languages, and as such, have little to no
support in modern \gls*{ide}s. To solve this, researchers need ad hoc solutions
for existing \gls*{ide}s, adding the needed functionality to test out their
language features. If this ad hoc solution is too extreme; outside the standard
functionality supported by the developers of the \gls*{ide}, it might be
short-lived. As the \gls*{ide} is maintained, updated and improved, the features
used to solve the niche needs of the experimental language might be deprecated.

However, if the \gls*{ide} has integrated support for extending the standard
functionality of the application, then the ad hoc solution will be more stable.
Such a system is known by many names. Plug-in architecture, extension \gls*{api},
or add-on system, to name a few. The common factor amongst these systems, is
that some component, be it a plug-in, an extension, an add-on, or a module, can
extend the functionality of the application. This is a modular approach to
extending the lifetime of an application; extending its software longevity. In
many of these systems, said components, are composable, allowing for multiple
components to work together in a modular fashion to add extra features to an
application. This way of adding functionality to an application is commonly used
in \gls*{ide}s.


\section{Modular architecture}

A zero-core, modular \gls*{ide} would assist in these ad hoc solutions. Even if
a new feature from an experimental language is introduced, it is unlikely that
this feature has no relation to existing features, and as such, it is easier to
extend the application in such a manner to facilitate this new feature, with help of
existing modules. However, if it is the case that this feature is
paradigm-shifting, then there will still be existing functionality that can be
used, re-used or extended to facilitate this.

\begin{hyp} \label{hyp:modular}
  When an application is designed to be modular from the start, then features
  not thought of, by the original developers can be integrated into the
  application, and be stable. If an experimental research language introduces
  some paradigm shifting concept, then this can easily be tested in a modular
  \gls*{ide}.
\end{hyp}

\section{Zero-core architecture}

Taking the modular architecture design to the extreme, the core application has
no base features, everything is enabled by an external module. We call such a
highly modular application, a \textit{zero-core} application. To qualify for a
\textit{zero-core} application, the default application has no functionality;
everything is acquired by modules. Such a design facilitates a modular approach,
enabling a module-developer to only focus on the functionality they want to
extend, not the entire core.

\section{Thesis outline}

Traditional \gls*{ide}s encompass essential features such as syntax highlighting,
code navigation, and hover-help, all of which play a crucial role in the
software development process. However, their limitations become apparent when
working with experimental languages. This paper advocates for modularization and
composability as key design principles, demonstrating their ability to extend
the operational lifespan of software by allowing for ease-of adoption to new
paradigms and tools. The discussion revolves around Magnolia, an experimental
research programming language developed by \gls*{bldl} at the University of
Bergen. Magnolia is a way to experiment with novel language features. It will
therefore be a case study illustrating the need for a specialized \gls*{ide}.

The focus point of this paper is to design a zero-core architecture, to develop
and implement a modular \gls*{ide}, where the target language will be Magnolia.

In chapter \ref{cha:background}, we will introduce Magonlia, and features this
language introduces that are difficult to encompass using standard \gls*{ide}s.
In chapter \ref{cha:ide} we will explore the use case of the zero-core, modular
\gls*{ide}, focusing on the different users of this application. Chapter
\ref{cha:impl} will discuss the design and implementation of the \gls*{ide},
mentioning different designs that were considered, some challenges that were
encountered, and the modules developed to add the necessary functionality to
qualify as an \gls*{ide}. In chapter \ref{cha:related}, we will discuss related
works, amongst them how different \gls*{ide}-vendors allow for extending of
their core application. Chapter \ref{cha:conclusion} will discuss the results
of our implementation, and answer our hypothesis \ref{hyp:modular}. Finally,
in chapter \ref{cha:future}, we will discuss the necessary work remaining.
