\chapter{Introduction}

Standard \gls{ide}s are indispensable tools in software development, offering features
like early bug reporting, project outline visualization, code highlighting, and
code completion However, these \gls{ide}s may not adequately support the unique
demands of experimental programming languages.
% Rewrite this
To solve this, a modular approach is used to designed and \gls{ide}.

\section{Experimental Languages}

Talk about Abstract Semantic Representation (ASR) Transformation,
Term Algebras, Mathematics of Arrays, and Syntactic Theory Functor (STF)

\section{Integrated Development Environment}

\todo{Rewrite this}
An \gls{ide}, aids a developer, as all the
needed tools for development are integrated into one application. There are
different kinds of \gls{ide}, generic and specialized.

A specialized \gls{ide} is one targeted towards a specific language, like Eclipse,
(reference?), or IntelliJ (reference?), which target Java/JVM. It contains
specialized features like:

\begin{itemize}
  \item Syntax Highlighting
  \item Code Autocompletion
  \item Go-to-definitions
  \item \dots
\end{itemize}

A lot of these features are possible due to \textit{Language Server Protocols},
which allow for a standardized way for compilers to give code-support to \gls{ide}'s.

\todo{Connect these better}

A generic \gls{ide} contains the features that are common among development in any
programming language, like:

\todo{Should write a paragraph or two about each of these points}
\begin{itemize}
  \item File explorer
  \item Project manager
  \item Version Control System integration
\end{itemize}
