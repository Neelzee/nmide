\chapter{Introduction}

Standard \gls{ide}s are indispensable tools in modern software development,
offering features like early bug reporting, project outline visualization, code
highlighting, and code completion, however, these \gls{ide}s may not adequately
support the unique demands of experimental programming languages. Experimental
languages could introduce new concepts like \gls{asr} Transformation, Term
Algebras, \gls{moa}, \gls{stf}, or other novel programming features. These are
concepts from the academic community, and are not common in \textit{mainstream}
languages, and as such, have little to no support in modern \gls{ide}s. To solve
this, researchers need ad hoc solutions for existing \gls{ide}s, adding the
needed functionality to test out their language features. If this ad hoc
solution is too extreme; outside of the standard functionality supported by the
developers of the application, then the solution might be short lived. As the
\gls{ide} is maintained, updated and improved, the features used to solve the
niche needs of the experimental language might be deprecated.

A modular \gls{ide} would assist in this. Even if a new feature from an
experimental language is introduced, it is unlikely that this feature has no
relation to existing features, and as such, it is easier to extend the
application in such a manner to facilitate this new feature, with help of
existing modules.
If, however it is the case that this feature is paradigm-shifting, then there
will still be existing functionality that can be used, re-used or extended to
facilitate this.

If, however, the \gls{ide} has integrated support for extending the standard
functionality of the application, then the ad hoc solution is more stable. Such
a system is known by many names. Plug-in architecture, exstension \gls{api}, or
plugin system, to name a few. The common factor amongst these system, is that
some component, be it a plug-in, an exstension, or a plugin, can extend the
functionality of the application. This is a modular approach to extending the
life time of an application; extending its software longevity. In many of these
systems, these components, are composable, allowing for multiple components to
work together in a modular fashion to add extra features to an application. This
is way of adding functionality to an application is commonly used in \gls{ide}s,
and can be taken to the extreme. If an application is designed to be modular
from the start, then features not thought of, by the original developers can be
integrated into the application, and be stable. If an experimental research
language introduces some paradigm shifting concept, then this can easily tested
in such a modular \gls{ide}. This will be the focus point of this paper,
designing, developing and implementing an modular \gls{ide}. The target language
will be Magnolia, an experimental research language being developed by
\gls{bldl} at the University of Bergen.

\section{Integrated Development Environment}

An \gls{ide}, aids a developer, as all the needed tools for development are
integrated into one application. There are two different kinds of \gls{ide}s,
generic and specialized. \todo{Source}

A specialized \gls{ide} is one targeted towards a specific language, like
Eclipse, (reference?), or IntelliJ (reference?), which target Java/JVM. It
contains specialized features like the following:

\paragraph{Syntax Highlighting} Highlighting important keywords, identifiers
and more, makes the language easier to read for the developer, allowing them to
spot easy to miss errors, like misspelling of keywords.

\paragraph{Code Autocompletion} Suggesting keywords, method names or even entire
code snippets, is a powerful tool an \gls{ide} can have. This is possible to
achieve, in some form, without being specialized, by for example, suggesting
text that already exist in the document, but is most useful if it is
specialized, and can suggest built-in methods. This allows a developer to not
having to remember exactly how methods are named, is the method to split a
string by some delimiter, \textit{split\_by} or \textit{split\_on}? As long as
the developer writes \textit{split}, the correct method name will be suggested.

\paragraph{Go-To-Definitions} Being able to quickly navigate to methods and read
their implementation is a useful tool for a developer, as less time has to be
spent navigating the project structure, to figure out where some method was
implemented, and more time can be spent actually developing.

A lot of these features are possible due to \textit{Language Server Protocols},
which allow for a standardized way for compilers to give code-support to
\gls{ide}'s.

A generic \gls{ide} contains the features that are common among development in
any programming language, like:

\paragraph{File Explorer} Most project nowadays is larger than one file, so
being able to visualize the project in a tree-like-structure, and navigate that,
is useful. This feature usually comes with the ability to manipulate the project
structure, by adding files, folders, moving files around, and deleting them.

But creating any \gls{ide} would still limit the lifetime of the application.
The best example of a long living active \gls{ide}, or, at least editor, is Vim
(source?). Vim is not a feature full editor, but it is simple, lightweight, and
works on any operating system. But most people use it, for how easy it is to
extend; Its lifetime has been greatly extended by the ease of modularization.
Any popular module for Vim is open-source, and therefore, if any module had an
active community around it, if the \textit{lead} developer of the module stopped
developing it, that community can continue to develop the module, either by
getting maintenance access to the repository, or by forking it. Ensuring the
lifetime of the module is extended.


\section{Language Server}

The most important features in a modern \gls{ide} are possible due to the
\gls{lsp}. \gls{lsp} is a protocol for a language server and editor,
(the client), with which they communicate, allowing for features like code
completion, syntax highlighting, marking of warnings and errors, as well as
refactoring routines. This client-server architecture, in \gls{ide}s are a
more recent development (2020s), as previously, support for the features
mentioned, were only possible due to specific modules being written for
specific \gls{ide}s. \gls{lsp} being the norm, is a sign of modularity being
preferred, as now a single \gls{lsp} can be created, and used across several
different applications, like IntelliJ, VS Code and Vim. While useful for
\textit{standard} language, this is the limiting factor when it comes to
supporting experimental languages, as not only does a new set of protocols need
to be appended to a language server, the editor itself needs to be changed to
actually use these protocols. This creates a lot of work, for both the \gls{ide}
developer and for the compiler developer. Here is where a modular approach can
help both. If some new functionality or feature is added to the experimental
language, this off course means the compiler/interpreter has to be expanded
and/or modified, but for the \gls{ide}, a module could be added/modified to
utilize this change, instead of having to change the entire application.

In this paper, we will start by discussing some background around modularity,
software longevity, and Magnolia (chapter 2). We will also discuss the different
ways to achieve modularity within an application, and point out different
behaviors that arise from such a setup (chapter 4). Finally, we will discuss the
results (chapter 5), and conclusions (chapter 6).
