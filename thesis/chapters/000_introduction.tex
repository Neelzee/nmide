\chapter{Introduction}
\todo{This is supposed to be a longer abstract}

Standard \gls{ide}s are indispensable tools in modern software development,
offering features like early bug reporting, project outline visualization, code
highlighting, and code completion, however, these \gls{ide}s may not adequately
support the unique demands of experimental programming languages. Experimental
languages could introduce new concepts like \gls{asr} Transformation, Term
Algebras, \gls{moa}, \gls{stf}, or other novel programming features. These are
concepts from the academic community, and are not common in \textit{mainstream}
languages, and as such, have little to no support in modern \gls{ide}s. To solve
this, researchers need ad hoc solutions for existing \gls{ide}s, adding the
needed functionality to test out their language features. If this ad hoc
solution is too extreme; outside of the standard functionality supported by the
developers of the application, then the solution might be short lived. As the
\gls{ide} is maintained, updated and improved, the features used to solve the
niche needs of the experimental language might be deprecated.

A modular \gls{ide} would assist in this. Even if a new feature from an
experimental language is introduced, it is unlikely that this feature has no
relation to existing features, and as such, it is easier to extend the
application in such a manner to facilitate this new feature, with help of
existing modules.
If, however it is the case that this feature is paradigm-shifting, then there
will still be existing functionality that can be used, re-used or extended to
facilitate this.

If, however, the \gls{ide} has integrated support for extending the standard
functionality of the application, then the ad hoc solution is more stable. Such
a system is known by many names. Plug-in architecture, exstension \gls{api}, or
plugin system, to name a few. The common factor amongst these system, is that
some component, be it a plug-in, an exstension, or a plugin, can extend the
functionality of the application. This is a modular approach to extending the
life time of an application; extending its software longevity. In many of these
systems, these components, are composable, allowing for multiple components to
work together in a modular fashion to add extra features to an application. This
is way of adding functionality to an application is commonly used in \gls{ide}s,
and can be taken to the extreme. If an application is designed to be modular
from the start, then features not thought of, by the original developers can be
integrated into the application, and be stable. If an experimental research
language introduces some paradigm shifting concept, then this can easily tested
in such a modular \gls{ide}. This will be the focus point of this paper,
designing, developing and implementing an modular \gls{ide}. The target language
will be Magnolia, an experimental research language being developed by
\gls{bldl} at the University of Bergen.
