\chapter{Magnolia IDE}

This section will discuss the new Magnolia \gls{ide}, the different users of the
\gls{ide}, and their possible experiences which was under consideration when
developing the application.

\section{User Perspectives}

This application has to consider different users

\subsection{Module Developer}

Being a zero core application; all functionality comes from modules, the module
developer experience is the most important.

\subsubsection{Language Agnostic Modules}

The largest limiting factor in module oriented applications, is the
\textit{language barrier} Most applications limit what language one can extend
an application with, like in "Visual Studio Code", where its
JavaScript/HTML/CSS. Or IntelliJ, where one can use Java or Kotlin. But what
does language agnostic mean in the context of programming languages? It is, and
always will be C. Any \textit{serious} language has some form of C-\gls{abi}.
So, for a module application to be Language Agnostic, it must be able to invoke
C Modules. This functionality can come from a module, creating a structure as
shown in figure. \todo{Create this figure}

This module could be a singular one, acting as a translator, translating the
data flowing between the core and foreign-modules, or each foreign-module could
have their own translator.

\subsubsection{Existing Third Party Libraries}

You can use existing JavaScript libraries

\subsection{IDE User}

Something-something compiled/runtime modules

\subsubsection{Standard User}

Stupid/Lazy

\subsubsection{Experienced User}

Stupid/Opinionated

\subsection{Maintainer}

Stupid/Lazy/Opinionated
