\chapter{Future Work}

Would've, could've, should've.


\section{Technical Debt}

It just piles and piles and piles on.

\subsection{Module Market}

No module market. How should we distribute modules?

\subsection{Testing}

No unit tests for the TypeScript side, no integration testing between the
frontend and backend.

\subsection{Attribute and Instructions}

Can't remove or change eventListeners currently. This is because to remove an
EventListener, the exact same function passed to the \textit{addEventListener}
must be used, which means a reference to this function needs to be stored, but
having two or more of the same type? It can get confusing for a module developer
of what should actually happen.

\subsection{Keypresses}

Create a module that emits keypresses, or maybe look into Tauri's
keybind/keypress thing.

\subsection{Inconsistent UI}

Difficult to keep the \gls{ui} representation consistent with the \gls{dom}.
\dots
A workaround to this was used, where depending on what element an
\textit{eventListener} was added to, the sent Event would be \textit{sticky},
meaning it would add extra arguments to the \textit{args} field of the Event,
like attribute information, id, value, etc. But this would not update the
\gls{ui} stored in the \gls{ide}, but rather give Modules a peek at the current
\gls{ui} state. A better solution would be to somehow keep track of \textit{all}
user interactions to the \gls{dom}, and somehow bubble these changes down to the
backend, where the \gls{ui} representation is managed.

\subsection{Compile-time Module Installation}

It's kinda cursed.

\subsection{Unify the tooling}

A lot of different scripts doing stuff, should be one \gls{cli}-tool.

\section{Modular Language Server}

Once the compiler \cite{wiig} is implemented, a proper \gls{ls} can be created.

\section{Modular Editor}

The prototype editor module develop for this \gls{ide} is subpar compared to
existing ones. A new one should be developed, in tandem with a \gls{ls} client.
This will ensure that this \gls{ide} can support more languages. This editor
should then utilize existing technology that is already used by other
\gls{ide}s, like the tree-sitter\footnote{\url{https://github.com/tree-sitter/tree-sitter}}
parsing system, which amongst other things, can help with syntax highlighting.
