\chapter{Related Work} \label{cha:related}

\section{Multi-way Dataflow Constraint System} \label{sec:mdcs}

One thing this application does not provide a solution for, is the difficulty
in designing good \gls{gui}. Following the \gls{mvc}-pattern, \gls{gui}s can
represent structures such as lists, which users might want to manipulate in
some fashion, like appending or rearringing the items in the list. Managing such
a change, especially one that involves \gls{gui} widgets can be a challenge,
since a change in the view should be reflected in the model, and encoding this
can be very involved. Luckily, there exists frameworks that makes this task
easier. \textit{WarmDrink}, \cite{warmDrink} \cite{dslMdcs} is a JavaScript
framework that allow a developer to declarative specify structural changes in
an application. \todo{Find out if this is actually possible}

Another issue in \gls{gui}s is optimizing performance in regards to events
triggered by user actions, such as scrolling, resizing or typing. These events
could happen many times in a second, while in theory user speed is trivial for a
computer to keep up with, there are instances where a module family could be
quite large, meaning many different modules are triggered by the same event many
times. There are techniques, called event coalescing, for handling this, like
debouncing and throttling.

\paragraph{Debouncing} Debouncing is a technique where you delay the sending of
an event until after some timeperiod $T$ has passed. Once the event is triggered
$T_0$ starts counting down. If the same event is retriggered while $T_0 > 0$,
$T_0$ is reset by $T_0 = T$. If $T_0 = 0$, then the event is sent. Ensuring that
$T$ is not too large, is important, as if $T$ is above some threshold, the user
of the \gls{gui} will notice, and it will make the application \textit{feel}
slow.

\paragraph{Throttling} Throttling is a similar technique to debouncing, except
instead of delaying the event by some time $T$, the event is only sent when
$T_0 = 0$. Meaning the event is sent at regular intervals, and could be sent at
the exact same point in time when the user triggered the event, or it could
happen at most, $T$ units after the user action.

Debouncing and throttling work in less complex \gls{gui} structures, but as the
amount of features in an application increases, the complexity will also
increase. These event coalescing stragegies are a source of subtle bugs, as
event coalecsing can easily break modularity. In a \gls{jsms}, this issue could
be solved by using \textit{flusable promises} \cite{flush}.

\paragraph{Automated Testing} Due to the exstensive modularity of the
application, all modules can be tested individually, by \textit{mocking} the
expected state and events. This means that breaking changes in one module can be
detected before End-To-End testing, which is expensive. \todo{Add source}
But this can only verify the general logic of a module and module family, not
the UI. To achieve such automation, one could rely on an automated testing
framework, like the one in \cite{autoUi}.

\paragraph{\gls{mdcs}} Since the data and the \gls{ui}
in the core application is separated, it allows for use of Multi-way Dataflow
Constraint Systems. Leveraging \gls{mdcs}-frameworks, like \textit{WarmDrink} as
mentioned in \cite{warmDrink} and \cite{dslMdcs}

\paragraph{JavaScript Module System} Due to the good compatibility with existing
JavaScript libraries, this zero core application can utilize things like
\textit{flushable promises} \cite{flush}

\paragraph{Expected \gls{ui} behaviour} When creating a \gls{ui}, it is
important that it behaves in a manner which the user expects \cite{leastGui}.

\paragraph{Rust Module System} Since the core application is made using Rust,
one could use a \gls{mdcs}-framework made for Rust \cite{mcdsRust}.

\paragraph{\gls{mdcs} tooling} Having good tooling for a framework is important,
as shown in \cite{toolMcds}


\section{Abstract Algebra}

Interesting stuff. Its cool to know the names of certain \textit{patterns} in
mathematics, which are useful in programming. Other people also think that,
which is why \textit{Algebraic Specification}, for specifying system behavior,
was popular in the 80s. You got stuff like \cite{cafeObj}.
