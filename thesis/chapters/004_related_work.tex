\chapter{Related Work} \label{cha:related}

\section{Existing module architectures in IDEs}

There exists module architectures in other \gls*{ide}s.


\subsection{Eclipse}

\gls*{ide}s are one of the most common application that supports extensions by
third-party code. \gls*{ide}s like Eclipse and IntelliJ are specialized for
working with Java, but they can still support other languages with the help of
modules. A module in Eclipse for instance, could extend Eclipse with
functionality like syntax highlighting, code completion, Go-to-definitions,
debugging, and more, for standard programming languages. A lot of this
functionality, comes from module-to-module extension, as in Eclipse modules can
extend modules, with the use of the \gls*{erpc}~\cite{eclipseRcp}.

\subsubsection{Eclipse Rich Client Platform}
\gls*{erpc} is a platform for building desktop applications. Eclipse being an
example of this platform in action. A plug-in could for example be responsible
for setting up the general \gls*{ui} layout, similar to our module,
\textit{ide\_framework}, and another plug-in could then modify this \gls*{ui} by
adding a file explorer, similar to our module \textit{ide\_explorer}.


\subsection{NetBeans}

An important part of NetBeans core architecture, is the NetBeans Module
\gls*{api}. This \gls{api} is responsible for supporting the
\textit{runtime environment}, which is the minimum amount of modules needed to
run the NetBeans application. A module in NetBeans is a JAR file, in which a
module can, amongst other things, list their public packages. This means that
other modules can directly invoke methods provided by this package.


\subsection{IntelliJ}

The extensibility IntelliJ has, is achieved by its extensions points
architecture. This is an \gls*{api} for plugins to integrate with the
\gls*{ide}. Plugins use this \gls*{api} to register their implementations, which
the \gls*{ide} then use. Between different versions of IntelliJ, plugins may be
broken, due to a breaking change in their plugin \gls*{api}.


\subsection{Visual Studio}

There are two kinds of extensions in Visual Studio, \textit{VSPackage} and
\textit{MEF} extensions. \textit{VSPackage} are mainly used to extend
functionality like tool windows and projects, while \textit{MEF} extensions are
used to customize the text-editor.

\subsection{Visual Studio Code (VS Code)}

Visual Studio Code is an extensible \gls*{ide}. It achieves this extensibility
with its extension \gls*{api}. This \gls*{api} allows for extensions to modify
the look and behavior of the \gls*{ide}. In fact, many of the core \gls*{ide}
features are possible due to built-in extensions.


\subsection{Vim}

In Vim, plugins are written using VimScript. All plugins located in Vims plugin
folder are loaded when Vim starts. This is where a user of Vim would place their
configuration plugins, adding custom commands, key binds, themes, and more. In
the ftplugin folder, plugins that are specific to file extensions are stored.
Assuming we have a \textit{python.vim} file in ftplugin, that file would be
loaded when we open a python file.

\subsubsection{Plugin management}

Most users use Vim with a plugin manager, a plugin, which manages other plugins.
These plugins can do mundane tasks like:

\begin{itemize}
  \item Download and install plugins from repositories
  \item Configure plugins
  \item Handle updates
  \item Handle enabling and disabling of plugins
\end{itemize}

\subsection{Emacs}

Emacs\footnote{\url{https://www.gnu.org/software/emacs/}} is an extensible text
editor. Almost all of Emacs functionality is achieved by writing code in Emacs
Lisp, a dialect of Lisp\footnote{\url{https://lisp-lang.org/}}. Unlike other
\gls*{ide}s, Emacs does not restrict modules to a certain \gls*{api}, instead
they are free to modify anything, as the functionality from the Emacs Lisp,
\textit{sits} atop of a core written in C, which abstracts away platform
specific code, and enables Emacs to be turned from an a \textit{simple} text
editor, to write and send emails, multimedia management and much more. Another
interesting thing about Emacs, is that all files are in \textit{buffers},
meaning the representation of a file shown to the user, is not necessarily the
contents of the file, some rendering is can be done by a plugin, before the file
contents is shown to the user.


\subsection{Theia}

\say{The Theia \gls*{ide} is a modern \gls*{ide} for cloud and desktop built on the Theia Platform}\footnote{\url{https://theia-ide.org/}}.

Eclipse Theia is a highly extensible \gls*{ide}, supporting \textit{extensions}
from VS Code, their own extensions and plugins, and \textit{headless} plugins.
Theia differentiates between extensions and plugins, where a plugin is installed
during runtime, and an extension is installed during compile-time.

Theia reuses components from VS Code, like their extensions \gls*{api}, which
enables them to support VS Code extensions. Theia plugins share similarities
with VS Code extensions, but are not restricted to \textit{just} the VS Code
\gls*{api}.

Theia's extensions are designed to add or remove existing core functionality in
Theia. They also have access to the entire core \gls*{api}.

An headless plugin runs without access to the frontend, meaning they are suited
for \gls*{cli} interactions, or similar use cases where a frontend is not
needed.

Theia is both a desktop \gls*{ide} and a web \gls*{ide}. Where there is no real
distinction between the two, both abstracting it to a frontend and backend. This
is similar to our zero-core \gls*{ide}, as we can make this abstraction due to
our usage of a web view for the desktop \gls*{ide}.


\section{Graphical User Interface development} \label{sec:guid}

A common complexity within application development, is \gls*{gui} development.
Using the \gls*{mvc}-pattern as an example, \gls*{gui}s can represent structures
such as lists, which users might want to manipulate in some fashion, like
appending or rearranging the items in the list. Managing such a change,
especially one that involves \gls*{gui} widgets can be a challenge, since a
change in the view should be reflected in the model, and encoding this can be
very involved.

Another issue in \gls*{gui}s is optimizing performance in regard to events
triggered by user actions, such as scrolling, resizing or typing. These events
could happen many times in a second, while in theory user speed is trivial for a
computer to keep up with, there are instances where a module family could be
quite large, meaning many different modules are triggered by the same event many
times. There are techniques, called event coalescing, for handling this, like
debouncing and throttling.

\paragraph{Debouncing} Debouncing is a technique where you delay the sending of
an event until after some time period $T$ has passed. Once the event is triggered
$T_0$ starts counting down. If the same event is re-triggered while $T_0 > 0$,
$T_0$ is reset by $T_0 = T$. If $T_0 = 0$, then the event is sent. Ensuring that
$T$ is not too large, is important, as if $T$ is above some threshold, the user
of the \gls*{gui} will notice, and it will make the application \textit{feel}
slow.

\paragraph{Throttling} Throttling is a similar technique to debouncing, except
instead of delaying the event by some time $T$, the event is only sent when
$T_0 = 0$. Meaning the event is sent at regular intervals, and could be sent at
the exact same point in time when the user triggered the event, or it could
happen at most, $T$ units after the user action.


\subsection{Flushable promises}

Debouncing and throttling work in less complex \gls*{gui} structures, but as the
amount of features in an application increases, the complexity will also
increase. These event-coalescing-strategies are a source of subtle bugs, as
event coalescing can easily break modularity. In a \gls*{jsms}, this issue could
be solved by using \textit{flushable promises}~\cite{flush}. This could have
solved our issue, where we had some event handler that took noticeably longer
time to return, but since this was a Rust module, we could \textit{solve} this
by doing this computation on another thread. If it was a JavaScript module we
could have solved it by using \textit{flushable promises}.

If we implement a \gls*{ls}-client in JavaScript, \textit{flushable promises}
could allow for a smoother experience, as things like \textit{looking up}
renaming in a Magnolia project is a more involved process for the compiler,
and in larger projects, could take a noticeably long time.


\subsection{Multi-way Dataflow Constraint System}

Luckily, there exists frameworks that make this task easier.
\textit{WarmDrink},~\cite{warmDrink, dslMdcs} is a JavaScript framework that
allow a developer to declarative specify structural changes in an application.
The framework can guarantee \gls*{gui} behavior, by utilizing a \gls*{mdcs},
ensuring it can create constraint systems. A constraint system is a
representation of how different variables are dependent on eachother. If one
variable is needed in the computation of another, they have a dependency
relation. By using a \gls*{dsl}, a developer can declare a constraint system.
But, similar to how working on our \textbf{Instruction}-set without tooling is
quite complex, tools have been created to ease the creation of such constraint
systems~\cite{toolMcds}. Given how WarmDrink is a JavaScript framework, it is
quite well suited for the web, aswell as our \gls*{ide}. A runtime system
specifically for a \gls*{mdcs} could be implemented for JavaScript modules.
Furthermore, there also exists a Rust implementation of this framework, which
means a similar system can be created for Rust modules.

\section{Automated testing}

Due to the extensive modularity of the application, all modules can be tested
individually, by \textit{mocking} the expected state and events. This means that
breaking changes in one module can be detected before \gls*{e2e} testing, which
is expensive. But this can only verify the general logic of a module and module
family, not the UI. To achieve such automation, one could rely on an automated
testing framework, like the one in~\cite{autoUi}. Or if one is working with a
\textit{simple} JavaScript runtime, one could use third party software like
\textit{Playwright} for creating tests, as it can auto generate the \gls*{dsl},
while the developer uses the module or entire \gls*{ide} if it is an \gls*{e2e}
test. This would help a module developer to discover behavior that a user might
not expect~\cite{leastGui}.


\section{Syntactic Theory Functor}

\gls*{stf} is a framework for creating, reusing and restructuring
specifications\cite{stf:haveraaen:2020}, specifications from algebraic
specification languages like CafeOBJ~\cite{cafeObj}. \gls*{stf}s are also used
by the new Magnolia compiler\footnote{As of May 2025, still in development}, to
resolve renaming in Magnolia and \textit{flattening} of the \gls*{asr} to be
shown to the developer \cite{wiig}.


\section{Abstract algebra}

Magnolia is a kind of algebraic specification language, like CafeOBJ~\cite{cafeObj}.
An algebraic specification language, is a language where one can develop
similarly as to how one might create an algebraic structure. We specify some set
and some functions that take elements from the set as inputs. Finally, we
specify the behavior of our functions, if they are associative, if there are any
predicates that the arguments to the functions need to fulfill, etc. As shown in
the development of this \gls*{ide}, this can be quite useful way of thinking.


\section{Language workbenches}

Language workbenches are tools for creation and use of computer languages~\cite{lwb}.
An \gls*{ide} is a kind of language workbench, as it can be used both for a
language, as described in this thesis, but also for creating languages, as this
too is a software project. More specific are the tools created by JetBrains,
Meta-Programming System (MPS)\footnote{\url{https://www.jetbrains.com/mps/}},
for creating programming languages, specifically \gls*{dsl}s. What makes tools
like MPS different from standard \gls*{ide}s, is that in standard \gls*{ide}s we
work on the source file of a program, while language workbenches work on the
\gls*{ast} of the program. Programming languages were made so that we
programmers could read closer to how we think, than how a computer thinks. So
before a computer can run our code, it needs to be translated. Generally, this
translation is done by parsing the source file into some tree-structure, and
then interpreting that tree, transforming it into an \gls*{ast}. As a concrete
example, in figure \ref{fig:instrTree} we have a visualization of an \gls*{ast},
for our \textbf{Instruction}-set.

So language workbenches then, enables a developer to work on this abstract
representation of the language one is creating.

Similarly, in Magnolia, it is useful for a developer to visualize the effect of
the compiler flattening the \gls*{asr}-tree, especially in regard to renaming.
An \gls*{asr}-tree is similar to an \gls*{ast}, except it has extra information
on each node.


\section{Language Server}

The most important features in a modern \gls*{ide} are possible due to the
\gls*{lsp}\footnote{\url{https://microsoft.github.io/language-server-protocol/specifications/lsp/3.17/specification/}}.
\gls*{lsp} is a protocol for a language server and editor, (the client), in
which they communicate, allowing for many of the features mentioned in section
\ref{sec:ide}, and explicitly mentioned in table \ref{tbl:ide}. \gls*{lsp} being
the standard since the 2020s, is a sign of modularity being preferred, as now a
single \gls*{lsp} can be created, and used across several different
applications, like IntelliJ, VS Code and Vim. While useful for
\textit{standard} language, this is the limiting factor when it comes to
supporting experimental languages, as not only does a new set of protocols need
to be appended to a language server, the editor itself needs to be changed to
actually use these protocols. This creates a lot of work, for both the
\gls*{ide} developer and for the compiler developer. Here is where a modular
approach can help both. If some new functionality or feature is added to the
experimental language, this off course means the compiler/interpreter has to be
expanded and/or modified, but for the \gls*{ide}, a module could be added and/or
modified to utilize this change, instead of having to change the entire
application.

\begin{table}[]
  \centering
  \caption{\gls*{ide} features enabled by \gls*{lsp}}
  \label{tbl:ide}
  \begin{tabular}{|l|l|}
    \hline
    IDE Feature & \gls*{lsp}-method \\ \hline
    Go to Declaration & textDocument/definition \\ \hline
    Go to Implementation & textDocument/implementation \\ \hline
    Auto-completion & textDocument/completion \\ \hline
    Hover & textDocument/hover \\ \hline
    Warnings & textDocument/publishDiagnostics \\ \hline
    Rename & textDocument/rename \\ \hline
  \end{tabular}
\end{table}

An example of this in action, say a developer is working on a file
\textit{main.ts}, in their Typescript project. They hover over a type imported
from, and defined in \textit{types.ts}. This is what happens:

\begin{enumerate}
  \item The editor detects the user is hovering over a \textit{special} word
  \item The editor sends a request to the Typescript \gls*{ls}
  \item The \gls*{ls} responds
  \item The editor formats the response into a small window showcasing the
    documentation and implementation of the type
\end{enumerate}
