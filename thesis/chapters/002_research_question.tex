\section{Research questions}

% TODO: Rewrite this
Exploring how modularization can facilitate the design and implementation of
an integrated development environment targeting experimental programming
languages.

% TODO: Red-thread is missing, find it

% TODO: Should probably also expand on what we mean by limiting (i.e. limiting
% in the sense of plugins being developed)

% TODO: Rewrite to be more academic enough

\paragraph{Language Agnostic} The largest limiting factor in module oriented
applications, is the \textit{language barrier} Most applications limit what
language one can extend an application with, like in "Visual Studio Code", where
it's \textit{just} JavaScript/HTML/CSS. Or IntelliJ, where one can use Java or
Kotlin. But what does language agnostic mean in the context of programming
languages? It is, and always will be (sorry Rust): C. Any language worth a damn,
can create some bindings with C. So, for a module application to be Language
Agnostic, it must be able to use C Plugins. This just means that the application
must be able to call function from a C library.

\begin{enumerate}
  \item Modular
  \item Open Sourced
    \begin{enumerate}
      \item Both for Plugins
      \item and for the Application
    \end{enumerate}
  \item Language Agnostic
  \item Easy to extend
  \item Easy to develop modules for
\end{enumerate}

% TODO: Expand these points
\begin{itemize}
  \item Software longevity
  \item Does modularization extend the lifetime of software?
  \item If not, what does?
\end{itemize}
