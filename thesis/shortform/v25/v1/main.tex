\documentclass[runningheads]{llncs}

\usepackage[T1]{fontenc}

\usepackage{graphicx}

\usepackage{newclude}

\usepackage{tikz}
\usetikzlibrary{shapes.geometric,positioning,shapes.symbols}

\title{Everyone is better at development than me\\
\large{Developing a modular IDE}
}

\author{Nils Michael Fitjar\inst{1}
}

\authorrunning{Nils Michael Fitjar}

\institute{University of Bergen
\email{nfi005@uib.no}
}

\begin{document}
\maketitle

\begin{abstract}
  This paper introduces a modular Integrated Development Environment (IDE) for
  experimental programming languages, addressing limitations in traditional
  IDEs. While standard IDEs are crucial in software development, their support
  for experimental languages is often inadequate, especially if these
  experimental languages introduce. This project proposes a modular IDE to
  extend its lifespan and enhance support for experimental languages. Analyzing
  the essential features of traditional IDEs and the need for adaptability to
  new paradigms and tools. Magnolia, a generic programming language developed at
  the University of Bergen, serves as a case study, highlighting its unique
  characteristics and the necessity for a specialized IDE.
  %% TODO: rewrite this part, as there are no modules which implements this.
  The primary research question explores how
  modularization facilitates the design and implementation of experimental
  programming languages. Specific modules tailored for experimental languages,
  including Abstract Semantic Representation (ASR) Transformation,
  Term Algebras, MoA translation, and Syntactic Theory Functor (STF), are
  outlined.
  %% TODO: Write about how modularity eventually leads to a zero-core-app
  \keywords{Modularization \and IDE \and Magnolia.}
\end{abstract}

\section{Introduction}

Standard Integrated Development Environments (IDEs) are indispensable tools in
software development, offering many features. However, these IDEs may not
adequately support the unique demands of experimental programming languages. An
example of this, is the research language Magnolia, being developed at the University
in Bergen, by the research group Bergen Language Design Laboratory. They have
developed an IDE for Magnolia, but this was in ???, so there is a need for an
new IDE.
%% TODO: Should I write about the tech-stack here? Unsure.

%% TODO: Should probably mention this in the introduction
Magnolia is
used by students at the university, so there is a need for an IDE
response, this project proposes a zero-core application solution.

\section{Background}

%% TODO: Rewrite, IDE's where made to solve the problem of setting up
%% development for different developers, ensuring they have similar
%% environments, to avoid trivial "it works on my machine, problems."
Traditional IDEs encompass essential features such as syntax highlighting, code
navigation, and hover-help, playing a crucial role in the software development
process. However, their limitations become apparent when working with
experimental languages.
%% TODO: Rewrite, maybe? Should mention something about software longevity.
The paper advocates for modularization and composability
as key design principles, demonstrating their ability to extend the operational
lifespan of software by adapting to new paradigms and tools. The discussion
revolves around Magnolia, a generic programming language developed at UIB, as a
case study illustrating the need for a specialized IDE. It is a way to
experiment with generic programming. And too achieve this in a sufficient
manner, a more specialized IDE is required.

\section{Research questions}

%% TODO: Should probably expand here?
Exploring how modularization can facilitate the design and implementation of
experimental programming languages.

%% TODO: Write about how the current IDE is obsolete, (is this the right word?)
%% Also mention why create an entire new IDE, instead of writing modules of an
%% existing one.
\section{Why develop an new IDE?}
The current Magnolia IDE was made with Eclipse, version ???, using now
deprecrated modules. %% TODO: List these modules

%% TODO: Write about current user experience, mention install times/complexity
\subsection{Things to learn from the current Magonlia IDE}
To install the IDE, you have to follow this guide:
%% TODO: Add the guide

Which sometimes works. But when teaching ???, two weeks is used to ensure all
students have access to the Magnolia IDE.

%% TODO: If there are any other known issues, list them here

So an IDE is needed to improve the IDE experience. But why modular?

%% TODO: Write about why the need for modularity, why not do what everyone else
%% is doing, and fork vscode and add AI?
\section{Why create an modular IDE?}
Why is an modular IDE is preferred over an \textit{regular} IDE? A lot of the
standard features
%% TODO: add footnote or something about how we will expand what features are
%% standard in an IDE, later.
in an IDE, cover the basic needs of Magnolia. So an exstension to an existing,
popular, IDE would cover some of the goals.
%% TODO: Somehow mention earlier, what the goals of this new IDE is.
But, while choosing something as popular like Visual Studio Code would be good,
since it contains a lot of the basic necessities for an IDE, and it will most
likely not deprecrate it's own modules\/functionality used in creating the new
%% TODO: Should probably make this point more clear.
Magnolia IDE, it cannot, support all the wanted features. %% TODO: Uhm source?

The best guarantee to not deprecrate functionality, or if deprecrating,
%% TODO: Would like to say "opensource", but vscode is opensource, so need to
%% find another way to say opensource.
replacing it, is to have an inhouse application.

%% TODO: Mention the uniqueness of Magnolia, which standard features like LSP
%% cannot totally fulfill.
\subsection{Challenges due to Magnolia}
%% TODO: Explain Magnolia

%% TODO: While unlikely, changing of Magnolias scope could happen, as it is
%% still under development, so features "hard-coded" into the application, could
%% be obsolete/outdated. If everything is "hard-coded", it will make it harder
%% for future developers to update/correct this. Having modularity as a first
%% -class feature, will mitigate this.
\subsection{Challenges due to possible changing scope}
Magnolia as a language, still under development, and all of the tooling that are
common in languages, are also under development. The compiler for Magnolia and
the IDE is being developed in parallel. So, while unlikely, there is the chance
of some \textit{breaking} change occurring in the future, which a developer
needs to keep in mind.

%% TODO: Write about what makes an IDE, an IDE.
\section{What is an IDE, really?}
What is an IDE? It is an integrated development environment; it is an
application, inwhich an user can develop, compile and run the project they are
working on. This is easier said than done.

%% TODO: Mention difficulties with getting developers started on new projects
%% due to missing environment variables, libraries or scripts needed to make
%% large projects work, and how IDE's fixed that problem.
\subsection{The solution to the problem}
%% TODO: Explain dynamic linking
If using a language like C, which uses dynamic linking, it is necessary to
ensure that all libraries are present. Before IDE's where a thing, this was
managed by using build-scripts, like CMake, but this can quickly get
complicated, especially onboarding new developers on an existing project, or
when developing, leading to the popular adage: \textit{it worked on my machine}.

This was solved by creating an application, inwhich the entire development
environment was a part of the application itself. Eclipse for example, allowed
for a lot of exstensibility, with its \textit{Plug-in Architecture}, which
allowed users to configure their IDE to their liking, with superficial features
like keybinds, themes, and similar, but also features aimed at improving the
developer experience, like syntax highlighting, compiling and execution of the
program available inside the IDE, and debugging.

\subsection{Expected Features}
Today, an IDE has a lot of functionality that are expected by a user.
%% TODO: Explain these features
\begin{enumerate}
  \item Syntax Highlighting
  \item Error stuff %% TODO: What is a good way to say: "shows me errors"
  \item Text edititing
  \item File Explorer
  \item Project management
  \item VCS integration
  \item Integrated Debugger
  \item LSP
  \item a module system
\end{enumerate}

%% TODO: Ensure that there is a good explanation for why modular design is used.
\section{Features}
%% TODO: Describe features that are available due to this modular design.
Going for a modular design, allows for:

%% TODO: Mention long-living-software like Vim, which are very modular.
%% And mention how future proofing is something all developers should
%% strive for.
\subsection{Future Proof}
As mentioned, one of the criteria for this new Magnolia IDE, is future proofing
it. In this context, future proofing an application, means ensuring longevity of
%% TODO: Rewrite this to "flow" better
%% TODO: Is it long living, or long-living?
it. This can be achieved by looking at existing long-living applications. An
application can be long-living due to many reasons, not all good. If it is the
only application that fulfills the role, it will be long-living due to
necessity, it is the only application that does what I need, so I have to use
it. A lot of legacy software falls under this category. Being the only
application that fulfills a role, does not make it good.

Another way for an application to be long-living, is if it is beloved by its
users. Applications like Vim falls under this category. One of the reason it is
 %% TODO: Uhm, source!?
beloved by users, is due to its excellent keybinds, called vim-motions. Infact,
it is so popular, that all other IDE's with a module system, has a module that
adds vim-motions. One can also argue that users enjoy using Vim because they can
configure it, exstensibly.
%% TODO: Write about how awesome the Vim modules are.

%% TODO: should this be here? I do not know, I have lost the red-thread.
Furthermore, a modular design approach suits well for projects who will have a
\textit{quick} rotation of developers. This being the result of a students
master thesis, most likely, the next person working on this application, will be
another student. Having a modular design philosophy to this project can ensure
faster development time in the future, given that each \textit{module} created,
are \textit{self-contained}, and do not \textit{bleed} into other modules. So,
if another developer wants to expand on the \textit{editor} module, they don not
have to understand what the \textit{file-explorer} module is doing.

%% TODO: Exstensibility is a wanted feature for advanced users, because reasons.
\subsection{Exstensible}

Exsensibility is cool, because then I, as a developer, can make all IDEs like
Vim, which is the editor I like the most.

%% TODO: Easy to install is necessary when technology-illiterate people use your
%% software.
\subsection{Easy to install}
%% TODO: This is a good point. How do I counter it? Make a module!
By being modular, the application is not necessary easy to install. While the
base application is minimal, it would still be a hassle to install all the
different modules needed to turn the application into an IDE.

%% TODO: Mention different module architecture, and how and why they fall short.
%% Especially when it comes to languages like Magnolia.
%% TODO: Mention the tech-stack here, maybe?
%% TODO: Is it module or module? Are there any differences? Can I use them
%% interchangeably, like I have been doing?
\section{Designing the Module Architecture}

%% TODO: Why did I choose a calling-module approach, instead of module-calling?
%% should also give examples about other IDEs/applications
When creating a modular application, there are two different ways to think about
the modules which will extend the application. Are they calling the application,
or is the application calling them?

%% TODO: Why did I end up choosing init-update-view setup? Should figure out a
%% way to properly integrate this, red-thread-wise, into this section.

Starting with Model-View-Controller, one has a good abstraction on how an
application behaves. There is some state, (the model), which is rendered as a
view, and then the user interacts with this view, changing the state.

%% TODO: Mention the first idea, maybe? It was not really well thought out, but
%% it is something to mention. I think it was something about a message-based
%% system, where modules would listen for messages and then do some action.
%% Like when you click this button, open a file.

This is a good starting point for a modular architecture. A module is called to
render the view, the view being everything inside the window of the application,
and when the user makes an interaction, the module is called. But where does the
state live? In the module? Or in the application? If the state lives in the
module, how does a module share their state with another one? Do they need to?
The more modular approach would be to allow interop between different modules.
%% TODO: Write this: Its like only being able to build around the core, or build
%% several different layers around the core. It can be messy; modules can bleed
%% out, if the module developers are not careful, but if done correctly it will
%% benefit the module developer more, to be able to do deep interop between
%% modules.
%% But in a smart way.

%% TODO: Find a way to signal, we ended up keeping the state in the module, but
%% coalecing it into a _super-state_ in the core. Allowing modules to share
%% states, kinda.

With this method, modules need initialize their own state, and update it. This
could be written as a singular method, as the result is just a state, but it is
a better module developer experience to strictly specify when the module is
first called, i.e. initialized, and when it is doing the \textit{n}th update.

%% TODO: Expand on this.
Now all that is left is the view. This, offcourse, depends on the state.

%% TODO: Is there a way to tie these two sections together?

So, we end up with something like this:
A module needs these three functions
\begin{itemize}
  \item init
  \item update
  \item view
\end{itemize}
Where init is called at the start of the application, to set the state, then
after all modules have been initialized, view is called, to set the view of the
application. Everytime a user interacts with the application, it updates the
state, which inturn updates the view.

%% TODO: Tie all this together, somehow.

To fully utilize the modular approach, a zero core IDE was preferable. In this
context, zero core, means the IDE has only the necessary functionality to load
modules, also called modules. This is an extreme approach, as this means
everything, meaning all functionality, needs to be supplied as a module. One
could create a mega-module, which encapsulates all needed features to qualify as
an IDE, but this would not fully utilize the features available by choosing a
modular approach.
To fully utilize the modular approach, one should keep in mind
\textit{granularity}, when designing modules. When designing an IDE there are
certain functionalities that can be reused by other modules. An example, in
an IDE like Visual Studio Code, there is something called an `file-explorer`.
%% TODO: Add picture of a file explorer
As shown in the picture above, it is a window which renders the project the
user is currently working on, as a tree-like structure, which enables the
user to quickly navigate between files. Similarly, most IDEs have indexed the
project they are working on, so that a user can quickly search for the files
available in the working directory. Both of these functionalities could be
supplied by the same module, or broken into three different ones.
%% TODO: Explain this better.
One for the file explorer, and one for searching for files. Both of which uses
the last module to figure out what to render, or what file to suggest.

Regardless if the module is a mega-module, or a family of smaller ones, a
question remains, how does a module work?

%% TODO: Mention why one wants a language agnostic module architecture
\subsection{Language Agnostic modules}
As this is a new application, it lacks modules. And, as mentioned, a good way
to ensure software longevity, is to ensure exstensibility, which modules are
needed for. A good way to ensure the existence of modules, is to either make
the scripting-language modules are made in, good, meaning the tooling around the
language is good. Or ensure the possible module developer can make the module in
their favourite language. This is not an easy feat to achieve, but it is
possible.
%% TODO: Should probably rewrite this, to sound smarter.
The easiest way to be technically language agnostic, is to allow for
modules made in C. As all \textit{serious} languages have bindings to C. So if
a module can be written in C, it is technically language agnostic. So this is
the way.

%% TODO: Why is pureness good?
\subsection{Pureness In modules}
The current specification of a module is vague enough to allow for some
restricting improvements. Among them being \textit{pureness}. Since a module is
called in easily determinable fashions, init first, then view, and then
everytime \textit{something} happens, update, and then view, repeating ad
infinitum. Introducing pureness opens up the possibility of many things.
Pureness, means that a module has no side-effect. A trivial example is the init
function of a module. Since it does not take any input, it is, or should be, a
constant function, the output is the same everytime. The same should hold for
the update and view function aswell.
%% TODO: Find a good way to explain why pureness is good without making
%% concrete examples, like with unit testing, E2E testing, and similar.

\subsection{The Tools}
As mentioned earlier, one of the goals of the new IDE is to allow for a
language agnostic module architecture, (LAPA), by allowing for modules to be
written in C. The easiest way to allow this, is to write the entire application
in C, as C can dynamically load a library during runtime.
%% TODO: Not for me, atleast
This is not feasible. But another low-level system language could help achieve
this. Rust is a programming language with a rich type system and ownership
model, that can guarantee memory-safety and thread-safety, both of which are
important factors when it comes to creating an application with module
capabilities. Using the Tauri framework, which enables application development
to be done both in Rust and JavaScript, for business logic and UI respectively,
one gets two languages for free; it allows for modules to be written in
JavaScript and Rust. And Rust being a low level system languages, has bindings
to both use, and be used by C libraries ensures a good starting point for the
LAPA.

%% TODO: Tie these two paragraphs better together

Using Tauri, allows the application to be divided into two parts. The
\textit{Frontend} and \textit{Backend}. The Frontend can be written in any
JavaScript framework, and the Backend is written in Rust. It uses a similar
concept as Electron, %% TODO: Uhm, sauce!?
where one basically creates a standard frontend-backend architecture, usually
keeping the business logic in the backend, and keeping the frontend concern to
just rendering the responses from the backend. Using \textit{crates} like Serde
%% TODO: Uhm, source?
structs can be easily serialized and deserialized to JSON, ensuring easy
communication between the Frontend and Backend.

%% TODO: Mention the TMap/TValue types created

Due to \textit{abi_stable}, there is no need to handle exceptions, the types are
always correct, so the results from calling foreign code is always valid. The
only possible way to have a \textit{panic}, is, as mentioned, having the module
be out-of-date. If this is due to a minor or major change on the
\textit{core-std-lib}, which is backwards compatibale, all that's needed to
correct this, is to recompile the module with the updated version.

%% TODO: This could probably be in another section, but I am just putting it
%% here
%% TODO: Maybe both `Rust module System` and `JavaScript module System` could be
%% sections under `Frontend` and `Backend`, that way I can more easily tie in
%% the technology stack, and all of the, "exposing", I've done on the Frontend.
\subsubsection{Rust module System}
This was the first module system designed. In this context, a module system is
the management of modules, loading, calling them, and handling exceptions or
errors. In Rust, this is not quite straight forward. One could write a Rust
module in standard Rust, with no problem, if the compiler used to compiler the
core and the module is on the same version, not same minor version, but same
patch version. %% TODO: Add some clarification about semver notation.
This is due to the fact that the Application Binary Interface, (ABI), in Rust
is not protected by the semver notation. That is to say, a stable ABI is not
something Rust guarantees.
This is because Rust does a lot of optimilizations on the layout of data in
memory. In C, for example, if one declare this struct:
%% TODO: Insert some C example of a struct with poorly optimized memory layout.

It will have this layout in memory:

%% TODO: Insert example of C struct in memory

It will have the same layout in memory. If, however, one creates a similar
struct in Rust:
%% TODO: Insert a Rust struct example, similar to the C one.

It will be optimized to this one:

%% TODO: Insert example of Rust struct in memory

Different memory optimalizations can occur on different patches of the Rust
compiler, RustC. Which means the ABI is not stable. To guarantee stability, one
can use other Rust libraries, called \textit{crates}, to achieve this.

%% TODO: Write about abi_stable and macros

With this implemented, the Rust module System, (RPS), can now be implemented.
A Rust project can be compiled to a C library, shared object file, which, by the
utilizes provided by \textit{abi_stable}, means we can safely load it. If the
%% TODO: Meantion the core-std-lib earlier.
module is out-of-date, meaning it compiled when the \textit{core-std-lib} had an
earlier version, it will simply error, instead of possibly causing undefined
behavior. If it is the correct version, it will be collected into a list of all
loaded modules, and whenever the core application is in init, view or update
state, the Rust modules will be called with the corresponding function, their
response mapped with the module name, (which is just the name of the file), and
returned to the Frontend, which is where the state is kept.

%% TODO: Rewrite this to better tie it in.
So this is the trivial example of a Rust module:
%% TODO: Insert the trivial Rust module
%% TODO: Make a note on how imports are left out. To make the example cleaner.

This section in update and view: `model: &RMap`, means that the function only
has a reference to the supplied model, effectively meaning only read access to
the model, since, for the update example, it only needs to return the field to
be updated, and not the entire model. This safes memory, in case the model is
very large, meaning it has many fields, we don't have to copy it everytime we
pass it to the module.

%% TODO: Write stuff
\subsubsection{JavaScript module System}
Since the Frontend is written in TypeScript, it is trivial to install and manage
JavaScript modules, (JSPs).

Here is the trivial JSP:
%% TODO: Insert the trivial JavaScript module
It is different from a Rust module, since as shown here:
%% TODO: Insert code for `window.modules`
It \textit{loads} itself. %% TODO: define the vernacular used.
Other than that, it is quite simple, for a module developer, atleast. For the
core it is more complex, since, loading and executing a JSP is more unsafe, than
that of RP.

\begin{enumerate}
  \item Throw an exception
  \item module can be invalid
  \item Data from the module can be invalid
\end{enumerate}

The first two points is solved by wrapping module invocations in
\textit{try-catch}, which at minimum, ensures an invalid module will not crash
the entire application. The last point is tricker, as doing it the naive way
could quickly lead to the entire core being full of nested if-statements like:
`if ("field" in obj)`. Luckily this can be avoided by using \textit{io-ts}, a
TypeScript library made to safely decode \textit{unknown} data received from a
third party.
Simply declare your expected type like this: %% TODO: Insert io-ts example

And use it like so: %% TODO: Insert decoding io-ts example

Which returns an `Validation<T>`, where `T` is the expected type. This, is a
wrapper around `Either<E, T>`, where `E` is a validation error. This validation
is also done on the results from the Backend.
%% TODO: Rewrite this
This is not strictly necessary, due to the strict type system Rust has, but
during development, it would catch issues with changing types, since, as
mentioned, the types are defined in the Backend.

%% TODO: JSPS and RPS are similar, but different.
\subsubsection{Differences in JSPS and RPS}
In theory, the JSPS and RPS are the same, but in practice, they are not. JSPs
have direct access to the DOM, and can change it outside of the standard module
cycle. This inturn would render the module impure, which voids some of the
guarantees the core gives.
%% TODO: Should I mention this? Feels like I kinda say my system is not good in
%% practice...
In practice, this does not matter. An example of this is a \textit{framework}
module. A framework module is a module who only creates a framework of HTML
which other modules can take advantage off.
%% TODO: This should be worded better
In the way rendering works, it would be tricky to design each module to render
exactly where they are supposed to be. This can be solved by doing this:
%% TODO: Add short example of removing a modules ability to call view.
Now, instead of calling other modules view, the view is only rendered when the
framework modules view is called, and when that happens:
%% TODO: Show example of framework-modules view

Using the \textit{class}-attribute, the framework module looks for classnames of
the variation `location-*`, where `*` should correspond to an \textit{id} of an
existing HTML element. Then the HTML is rendered as a child. This is not pure,
since the framework module renders HTML as a side-effect of calling `init`. But
this is a sacrifice to allow for easier time of rendering different modules as
components of an IDE.

%% TODO: What guarantees does the IDE have? What is voided when a user does not
%% uphold this contract?
%% TODO: Maybe talk about the guarantees before `Differences in JSPS and RPS`?
\subsection{Guarantees}
%% TODO: Is this it?
\begin{enumerate}
  \item Calling order
  \item DOM order
\end{enumerate}

%% TODO: Modularity is preferred over pureness, as pureness does not directly
%% affect software longevity.
\subsection{Modularity over pureness}
%% TODO: Kinda mentioned this in `Differences between JSPS and RPS`.

%% TODO: Mention how pureness could be achieved by adding the DOM or a DOM-like
%% object to the type signature to the module, which would mitigate some of the
%% differences between JSPS and RPS, but this would quickly lead to bloat.
\subsection{Optimalization over pureness}
%% TODO: I can mention optimalization, even though I have not done it, since by
%% exposing the functionality like I have, other modules could do this
%% optimalization.

%% TODO: Mention how when writing such an extreme modular application is created
%% it is no longer a specific application, but something that can be extended to
%% the wanted design.
\section{Modular Application}
%% TODO: Granularity is important here, as if the family of modules are granular
%% enough, it lowers the bar for turning the application from something that
%% solves A, to B. Especially if A and B are similar. An example could be
%% turning the IDE into an audio editor program, like Audacity.

%% TODO: Somehow get this into the thesis, because it is funny to mention X,
%% (formerly known as Twitter), because Elon is a dumbass.
%% Could be an excuse to mention how the application could "easily" be turned
%% into an web-based IDE, due to the tech-stack chosen.
\subsection{The \textit{everything} app}
%% TODO: Mention the modules which are made, which combined, give the everything
%% app the functionality needed to be considered an IDE.
\section {IDE-modules}
The easiest way to turn the everything app into a specific one, is by creating a
\textit{family} of modules. This means, a group of modules which work together
to give the necessary functionality to the application. Here is an example for a
basic IDE: %% TODO: Show a diagram of the IDE-module-Family

%% TODO: List the functionalities given by this module family.
\subsection{Features}

%% TODO: Rewrite this list
\begin{itemize}
  \item module Debugger
  \begin{itemize}
    \item Display State
    \item Display Msg's being sent
    \item Check if there are state collisions
    \item Check if there are Msg collisions
  \end{itemize}
  \item Dependency Viewer
  \begin{itemize}
    \item Showcase JS
  \end{itemize}
  \item File Explorer
  \item Text Editor
  \begin{itemize}
    \item Use existing JS
  \end{itemize}
  \item Rendering Framework
  \begin{itemize}
    \item Facilitate ease of pre-DOM-manipulation
  \end{itemize}
\end{itemize}

%% TODO: Mention how granularity helps a module developer develop useful
%% modules, as granularity is an important concept in modular design.
%% I guess I could find litreature in modular design.
\subsection{Granularity}
%% TODO: Kinda mentioned this already, if I need to write more on this topic, it
%% should be on a section earlier.

%% TODO: Funny to have this as a title, but the idea behind it is to mention how
%% while developing this application, I have had different "hats"; kept
%% different kinds of users in mind. But luckily, due to the modularity of my
%% design, it was easy to switch context.
%% TODO: This section should probably be earlier.
\section{Different hats}
When developing a zero core modular IDE, there are several different user bases
to keep track off.

\begin{enumerate}
  \item The developer who is going to use the IDE
  \item The module developer who is going to create the modules for the application
  \item The developer who is going to maintain/expand the core application
\end{enumerate}

\subsection{The Developer Experience}
\subsection{The module Developer Experience}
\subsection{User Experience Challenges}

\end{document}
