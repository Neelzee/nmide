\documentclass[12pt,oneside,final,a4paper]{report}

\usepackage{../../generators/imports}

\title{Everyone is better at development than me\\
\large{Developing a modular IDE}
}

\author{Nils Michael Fitjar\inst{1}
}

\authorrunning{Nils Michael Fitjar}

\institute{University of Bergen
\email{nfi005@uib.no}
}

\begin{document}
\maketitle

\begin{abstract}
  This paper introduces a modular Integrated Development Environment (IDE) for
  experimental programming languages, addressing limitations in traditional
  IDEs. While standard IDEs are crucial in software development, their support
  for experimental languages is often inadequate, especially if these
  experimental languages introduce. This project proposes a modular IDE to
  extend its lifespan and enhance support for experimental languages. Analyzing
  the essential features of traditional IDEs and the need for adaptability to
  new paradigms and tools. Magnolia, a generic programming language developed at
  the University of Bergen, serves as a case study, highlighting its unique
  characteristics and the necessity for a specialized IDE.
  %% TODO: rewrite this part, as there are no modules which implements this.
  The primary research question explores how
  modularization facilitates the design and implementation of experimental
  programming languages. Specific modules tailored for experimental languages,
  including Abstract Semantic Representation (ASR) Transformation,
  Term Algebras, MoA translation, and Syntactic Theory Functor (STF), are
  outlined.
  %% TODO: Write about how modularity eventually leads to a zero-core-app
  \keywords{Modularization \and IDE \and Magnolia.}
\end{abstract}

\chapter{Introduction}

\section{Experimental Languages}

\section{Integrated Development Environments}

\section{Modularity}

\chapter{Background}

\section{IDE History}

\section{Magnolia}

\subsection{Magnolia Stuff}

\section{Current Magnolia IDE}

\chapter{Research Question}

\section{???}

\chapter{Research method and evaluation}

\section{???}

\chapter{Engineering method}

\section{Modularity}

\subsection{Granularity}

\subsection{Module families}

\section{Tech stack}

\chapter{Results}

\section{Module 1}

\subsection{Pros \& cons}

\section{Module 2}

\subsection{Types}

\subsection{Pureness}

\subsection{Collisions}

\subsection{Pros \& cons}

\section{Module 3}

\subsection{Types}

\subsection{Tree Manipulation}

\subsection{Pros \& cons}

\chapter{Conclusion}

\section{Further work}

\end{document}
