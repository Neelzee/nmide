\documentclass[runningheads]{llncs}

\usepackage[T1]{fontenc}

\usepackage{graphicx}

\usepackage{newclude}

\usepackage{tikz}
\usetikzlibrary{shapes.geometric,positioning,shapes.symbols}

\title{Everyone Is Better At Development Than Me\\
\large{Developing Nn Modular IDE}
}

\author{Nils Michael Fitjar\inst{1}
}

\authorrunning{Nils Michael Fitjar}

\institute{University of Bergen
\email{nfi005@uib.no}
}

\begin{document}
\maketitle

\begin{abstract}
  This paper introduces a modular Integrated Development Environment (IDE) for
  experimental programming languages, addressing limitations in traditional
  IDEs. While standard IDEs are crucial in software development, their support
  for experimental languages is often inadequate, especially if these
  experimental languages introduce. This project proposes a modular IDE to
  extend its lifespan and enhance support for experimental languages. Analyzing
  the essential features of traditional IDEs and the need for adaptability to
  new paradigms and tools. Magnolia, a generic programming language developed at
  the University of Bergen, serves as a case study, highlighting its unique
  characteristics and the necessity for a specialized IDE.
  %% TODO: rewrite this part, as there are no plugins which implements this.
  The primary research question explores how
  modularization facilitates the design and implementation of experimental
  programming languages. Specific modules tailored for experimental languages,
  including Abstract Semantic Representation (ASR) Transformation,
  Term Algebras, MoA translation, and Syntactic Theory Functor (STF), are
  outlined.
  %% TODO: Write about how modularity eventually leads to a zero-core-app
  \keywords{Modularization \and IDE \and Magnolia.}
\end{abstract}

\section{Introduction}

Standard Integrated Development Environments (IDEs) are indispensable tools in
software development, offering many features. However, these IDEs may not
adequately support the unique demands of experimental programming languages. An
example of this, is the research language Magnolia, being developed at the University
in Bergen, by the research group Bergen Language Design Laboratory. They have
developed an IDE for Magnolia, but this was in ???, so there is a need for an
new IDE.
%% TODO: Should I write about the tech-stack here? Unsure.

%% TODO: Should probably mention this in the introduction
Magnolia is
used by students at the university, so there is a need for an IDE
response, this project proposes a zero-core application solution.

\section{Background}

%% TODO: Rewrite, IDE's where made to solve the problem of setting up
%% development for different developers, ensuring they have similar
%% environments, to avoid trivial "it works on my machine, problems."
Traditional IDEs encompass essential features such as syntax highlighting, code
navigation, and hover-help, playing a crucial role in the software development
process. However, their limitations become apparent when working with
experimental languages.
%% TODO: Rewrite, maybe? Should mention something about software longevity.
The paper advocates for modularization and composability
as key design principles, demonstrating their ability to extend the operational
lifespan of software by adapting to new paradigms and tools. The discussion
revolves around Magnolia, a generic programming language developed at UIB, as a
case study illustrating the need for a specialized IDE. It is a way to
experiment with generic programming. And too achieve this in a sufficient
manner, a more specialized IDE is required.

\section{Research questions}

%% TODO: Should probably expand here?
Exploring how modularization can facilitate the design and implementation of
experimental programming languages.

%% TODO: Write about how the current IDE is obsolete, (is this the right word?)
%% Also mention why create an entire new IDE, instead of writing plugins of an
%% existing one.
\section{Why develop an new IDE?}

%% TODO: Write about current user experience, mention install times/complexity
\section{Things to learn from the current Magonlia IDE}

%% TODO: Write about why the need for modularity, why not do what everyone else
%% is doing, and fork vscode and add AI?
\section{Why create an modular IDE?}

%% TODO: Mention the uniqueness of Magnolia, which standard features like LSP
%% cannot totally fulfill.
\subsection{Challenges due to Magnolia}

%% TODO: While unlikely, changing of Magnolias scope could happen, as it is
%% still under development, so features "hard-coded" into the application, could
%% be obsolete/outdated. If everything is "hard-coded", it will make it harder
%% for future developers to update/correct this. Having modularity as a first
%% -class feature, will mitigate this.
\subsection{Challenges due to possible changing scope}

%% TODO: Write about what makes an IDE, an IDE.
\section{What is an IDE, really?}
%% TODO: Mention difficulties with getting developers started on new projects
%% due to missing environment variables, libraries or scripts needed to make
%% large projects work, and how IDE's fixed that problem.
\subsection{The solution to the problem}
%% TODO: Mention how before woke, a developer only needed two things.
%% 1. An editor.
%% 2. A compiler.
\subsection{Expected Features}

%% TODO: Describe features that are available due to this modular design.
\section{Features}
%% TODO: Mention long-living-software like Vim, which are very modular.
%% And mention how future proofing is something all developers should
%% strive for.
\subsection{Future Proof}
%% TODO: Exstensibility is a wanted feature for advanced users, because reasons.
\subsection{Exstensible}
%% TODO: Easy to install is necessary when technology-illiterate people use your
%% software.
\subsection{Easy to install}

%% TODO: Mention different plugin architecture, and how and why they fall short.
%% Especially when it comes to languages like Magnolia.
%% TODO: Mention the tech-stack here, maybe?
\section{Plugin Architecture}
%% TODO: Mention why one wants a language agnostic plugin architecture
\subsection{Language Agnostic Plugins}
%% TODO: Why is pureness good?
\subsection{Pureness in Plugins}
%% TODO: JSPS and RPS are similar, but different.
\subsubsection{Differences in JSPS and RPS}
%% TODO: What guarantees does the IDE have? What is voided when a user does not
%% uphold this contract?
\subsection{Guarantees}
%% TODO: Modularity is preferred over pureness, as pureness does not directly
%% affect software longevity.
\subsection{Modularity over pureness}
%% TODO: Mention how pureness could be achieved by adding the DOM or a DOM-like
%% object to the type signature to the plugin, which would mitigate some of the
%% differences between JSPS and RPS, but this would quickly lead to bloat.
\subsection{Optimalization over pureness}

%% TODO: Mention how when writing such an extreme modular application is created
%% it is no longer a specific application, but something that can be extended to
%% the wanted design.
\section{Modular Application}
%% TODO: Somehow get this into the thesis, because it is funny to mention X,
%% (formerly known as Twitter), because Elon is a dumbass.
%% Could be an excuse to mention how the application could "easily" be turned
%% into an web-based IDE, due to the tech-stack chosen.
\subsection{The everything app}
%% TODO: Mention the plugins which are made, which combined, give the everything
%% app the functionality needed to be considered an IDE.
\section {IDE-Plugins}
%% TODO: List the functionalities given by this plugin family.
\subsection{Features}
%% TODO: Mention how granularity helps a plugin developer develop useful
%% plugins, as granularity is an important concept in modular design.
%% I guess I could find litreature in modular design.
\subsection{Granularity}

%% TODO: Funny to have this as a title, but the idea behind it is to mention how
%% while developing this application, I have had different "hats"; kept
%% different kinds of users in mind. But luckily, due to the modularity of my
%% design, it was easy to switch context.
\section{Different hats}
\subsection{The Developer Experience}
\subsection{The Plugin Developer Experience}
\subsection{User Experience Challenges}

\end{document}
