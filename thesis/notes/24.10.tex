\section{24.10.03}

I can't for the life of me do JS-Plugin support, shit sucks.
Should I ditch React and just do TS/JS? IDK, man.


\section{24.10.09}

I need a View for the frontend, and one for the backend, and finally, one for
coalescing both of them.


\section{24.10.10}

I'll drop first-class-plugin support for JavaScript, and instead just allow
plugins to create script-tags, which can do whatever. I can reuse the asset
logic I've used when I attempted the full support of JavaScript, but there are
a lot of cooler code-samples I can't use, but hopefully write a little about,
since I've fallen for functional TypeScript. It is what it is.

I think my commit history will be fucked. Since I've been working on a branch
with the idea of squashing commits, and therefore ignoring standard commit
message etiquette.


\section{24.10.11}

I should've made tests way earlier, small shit doesn't work, that should work,
which makes debugging the larger shit harder, because it could be due to the
small shit not working. Shit.


\section{24.10.19}

I hope that I can actually do JavaScript plugins now, since I have finally made
tests, and doc-tests, (cool Rust-feature). I think I should split the HTML-part
into a frontend and backend one, and just coalesce them before rendering. That
way I can do some caching or diffing to ensure I don't needlessly re-render the
view.
At least all the Rust-plugins work, so I hope I can design the
JavaScript-Plugin-System, (JSPS?), around it, without breaking it. I hope I can
write about this, but I should probably create more diagrams to display, to pad
my thesis, both with the diagrams, and also paragraphs explaining the diagrams
and the general idea behind them.

I think I can get this to work, I have successfully \textit{installed} a
JS-Plugin, now I need to call init on it, and merge the states. Hopefully this
will be easier.


\section{24.10.20}

I have a new problem. Well, kinda and also not really. With the JSPS support, I
have introduced race-conditions to my state change. I feel like it's more of a
problem I can drag out in writing, than an actual problem to solve with code.
Practically, it should not be a problem, just use FIFO. If, for some reason, two
Msg are arriving at the same time, we have two cases. It's either A: a
coincidence, due to the \textit{pureness} of plugins, it should not matter to
the state if one of them comes in first. Or B: the plugin-developer has made a
\textit{mistake}, as two different Msg being sent at the same time, which does
have an overlapping effect on the state, should probably not happen.
But we'll see if this is actually the case when I actually begin developing
plugins.

I have issues with race-conditions, regardless. I have a lot of different
effects that need to run synchronously, but run asynchronously.


\section{24.10.24}

I think I can solve JSPS stuff by using promises. Insane concept, I know, but I
think my issue has been with how React re-renders stuff. Maybe it doesn't
\textit{kill} the asynchronous function that is in the useEffect, when it
re-renders the component, I do not really know.

I like the way Promise-chaining looks, but I can't get it to work, I need to
redesign the way my architecture looks, to allow this, which probably means
more frontend control. Is it bad that I spend most of the time designing the
architecture? Hopefully not, because a good foundation should facilitate quick
development.

I've also thought about how to use these notes, I think I can feed them to the
garbage machine creator, (AI), since it's always garbage out, it's no problem
feeding it garbage. It might summarize my notes for me better than I could.


\section{24.10.30}

Magnolia Workshop. What is most relevant to me, is compiler-IDE-communication.
Can we use some LSP from the compiler? For compiling the code, CLI is the
standard.

Plugins should be installed by the backend.
