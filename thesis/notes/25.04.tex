\section{25.03.02}

Wasted the whole day, trying to build the rust code using flakes. Did not get it
to work, and noticed I don't get any LSP help, probably due to linking issues. I
think the best solution is to distro-jump to Debian, which \textit{just works},
and add Nix as a package which I could use for development. Getting stuff to
work in on NixOs is hard, probably in some part due to me not reading the
documentation that well. Or being able to find the documentation.
Regardless, there is some restructuring in the code that is ahead of me, I need
to be finished with the implementation stuff in two weeks, at the end of April
at the latest.

I think I'll just write some stuff, and hope it works, code-wise, until I get a
proper environment setup.

\section{25.03.03}

Got stuff to work, back on Debian again. I think I should just rewrite the base
to use \textit{regular} Rust stuff, so none of the ABI stuff. This will simplify
the code, while still allowing me to keep the stuff I've written about, since I
can talk about \textit{it would be cool if \dots}, instead of having to actually
implement the cool stuff.

\section{25.03.04}

I have discussed this before, I think, but how should the frontend and backend
communicate? I am leaning towards a frontend \to backend communication method,
so the frontend does the rendering, and event \to event mapping, but the backend
does all the state management. But will look into it.

I've done some refactoring, again. I have Rust Traits now, which I've used to
create an API for my modules, which other foreign languages can implement. I
will have to send UI modifications to the frontend, but I could coalesce them
all in the backend, by finding UI modifications that "just" result in the
identity.

Doing the modification/instruction thing is difficult, at least the way I want
to implement it. I wanted it to be some kind of DSL, so that I could analyze it.
But this is proving difficult. I think it is because I am over-engineering it,
so instead of having "add node to node", or something like that, I am supplying
a function which does the modification. It is probably better if I transpile the
DSL to the function, after I've done some analysis on it, to remove redundant or
invalid instructions.

\section{25.03.05}

It is difficult to do something correct. Not that I think I am doing it
correctly, but it is difficult. I've made some mix of a DSL + function
supplying. Which is a combination of what I said I was going to do, with what I
said I was not going to do. I think it works, it uses Rust Traits to encode
field modifications on a type-level, which is pretty interesting. I just have to
make a struct for each field that is modifiable, supplying a setter and getter.

Just setting the state is okay, the \textit{only} issue is reporting to
the module developer that the field does not exist, or something like that if
they try to modify a non-existing field. Which probably can be done
\textit{easily}. I think making an instruction set for each tree is the way to
go, at least until I see a way to combine them later.

\section{25.03.06}

Stuff works, kinda, so that's fun. I am having issues with the Trait system, as,
surprisingly, doing runtime-thread Trait usage is difficult. It made it easier
to integrated Modules, as I could simply implement the interface, but maybe
looking into using a struct instead? I would have a similar problem, where I
have field for a function, but I don't really know. Tomorrow, I'll look into
the async_trait crate, which apparently can help here.

\section{25.03.07}

Stuff works again, took only 4 months. Well, it kinda works. Still some kinks to
iron out, but should be able to figure out that by the end of the week. I should
definitively mention the trait system in Rust, because I am using it a lot. I
should also re-write the API section, to be more story like, implementing
logging and then the editor, in Rust and Magnolia. Looking at the differences,
especially how refactoring it to use semigroup and group traits/concepts.

I think I am going to remove everything I am not using, and either re-write it
if I need it, or checking previous commits to find it again. Since there is so
much garbage here currently, it probably feels good for the repository to have
commits which is just "deleted file".
