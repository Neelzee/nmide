\section{24.12.01}

Using Monaco will be more difficult that I first imagined. And by that, I mean
I can't simply wrap a trivial plugin around the library, and then use it. There
are some extra functionality/configurations that are needed, which I do not
know. Hopefully I'll figure this out, but I'll postpone this work until later,
since I'd rather not mess around with that this year. But I probably will. I
should create extensive documentation on the core IDE functionality, so that I
have something to write about, and to see if I have thought of everything.
Having to explain everything, will make it easier to see if I have made the
right decision, or just a decision.

I think today, I'll quickly look at turning the IDE into a server

\section{24.12.07}

I've almost turned the IDE into a server, got some CORS-issues, which as usual
is the bane of my existence. Fixing it should be easy, but I don't think I will
prioritize it in the coming days. I learned that after I am done with my
education at UIB, they will delete my account, and therefore my repository. So
I think as part of my redo/fix my CI/CD-job, I'll also look into migrating from
GitLab to GitHub. Once that has been completed, I need to plan the IDE-Plugins
I am going to make, I need a plan, not only to make the development easier, but
also so, in case I do not have anything substantial to showcase, I can instead
say, oh this? This is just a prototype of my awesome plan.

\section{24.12.11}

For the rest of the week, I need to figure out what features my IDE-Plugin
-Ecosystem are going to offer. The idea behind IDE-Plugin-Ecosystem, are the
plugins needed to turn the application into an IDE. I would like for the
majority of these plugins to be pure, but we'll see what happens.

There is a lot of functionality that is missing on the Rust side, that exist on
the JSPS side. This is mostly due to the lack of plugin development happening in
Rust, and will probably not catch up. This is because debugging JSP is easier
than Rust Plugins, since I can just add a console.log in the Plugin, and then
restart the application. If I were to do something similar in Rust, I would
have to re-compile the plugin, move it to the plugin folder, and then restart
the application. I think the only place the Rust Plugin Experience could ever
excel, is the writing of Models/Html, by utilizing macros. But developing such
macros would take more time than I am currently willing to invest.

\section{24.12.13}

I need to figure out what unique features Magnolia has, and by unique, I mean
concepts that are not supported under the LSP-specification. I know some stuff,
like renaming, but I do not have a concrete grasp on what is needed to enhance
the developer experience. But I do have some ideas about general features for
enhancing the developer experience.

\section{24.12.15}

I have a general outline for my thesis.

\begin{enumerate}
  \item Why create an IDE
  \item Things to learn from previous Magnolia IDE
  \item Why create a modular IDE?
  \begin{enumerate}
    \item Challenges due to Magnolia
    \item Challenges due to possible changing scope
  \end{enumerate}
  \item Features
  \begin{enumerate}
    \item Future Proof
    \item Exstensible
    \item Easy to install
  \end{enumerate}
  \item Plugin Architecture
  \begin{enumerate}
    \item Language Agnostic Plugins
    \item Pureness in Plugins
    \begin{enumerate}
      \item Differences in JSPS and RPS
    \end{enumerate}
    \item Modularity over pureness
    \item Optimalization over pureness
  \end{enumerate}
  \item Modular Application
    \begin{enumerate}
      \item The everything app
    \end{enumerate}
  \item IDE-Plugins
    \begin{enumerate}
      \item Features
      \item Granularity
    \end{enumerate}
  \item The Developer Experience
  \item The Plugin Developer Experience
  \item User Experience Challenges
\end{enumerate}

I think this is a good start, I just need to find out where to stuff references
to other literature

\section{24.12.20}

I can rework my entire plugin handling system. Because, it's all the same. I
call a plugin, give it arguments, (or not), carch any errors, decode the output,
ensuring it is valid, and then return it, or an error. Especially for init and
update, I can use the exact same logic for handling collisions, but I can also
include view.

\section{24.12.21}

There are still improvements needed, since there is some duplicated code on init
and view, but this can wait. Plugins can now add a function to turn a possible
collision into an allowed one, by fields. They get this input:
[Either ([(string, TMap)], string) (string, TMap)], which is a list of Either
a collision, which is a tuple, where the first element is a list of plugins and
their corresponding model, and the second element is the field they collided on.
The function, is a partial mapping, so only the wanted values are kept. The
default is to drop all `Left` values. There is still the possibility of a
collision occurring post coalecing, as if two different plugins work on the same
field, they both return two states that can collide. Currently, this is
mitigated by keeping the first instance, i.e. the first state found, this can
probably be improved.
