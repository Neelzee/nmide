\section{24.09.04}

Created my own Gitlab runner. Out of an old laptop. It did not work out of the box, but I hope
I can get it work by the end of today.

(1 minute later)

I dont know why, but my runner is disabled for 1 hour. Cool beans.

\dots

I re-installed debian on the old laptop, and registered the runner on the laptop, instead of
through SSH, but this did not work either. I also added a runner on my current laptop, and
this worked, somehow. But it was a flakey process. I saw on some forums that I should run
gitlab-runner with sudo, but this did not work on my machine, neither did without, it was
just stuck, it seemed. I finally ran it with the --debug flag, and then it worked.

I can now see that downloading 2G might be slower than just building it.

The pipelines take a ridicolus amount of time to run, so I should look into splitting them up
there is no need to run them for every little commit. This is a tomorrow me's problem, however.
I am also noticing that my desktop is lagging when being used as a runner, so I should enable
the shared runners for those jobs that it can run (by using tags).

I also need to check out the test coverage being created for nmide-core, it is not working for
some reason.


\section{24.09.05}

I forgot to turn on my gitlab-runner before going to JAFU, so now I can't build unless I create
a new gitlab-runner. I also can't ssh into my old laptop, because I don't know the ip, nor if
that port is open on our network.

So no ci work today.

I know of other problems I can/should work on, like the Rust ABI. It is not stable. Which is a
major problem. The Rust Compiler does a lot of optimalizations on data-types, which means across
different semver's, the same struct can have different orientation in memory, which leads to
undefined behavior. There are crates that can fix this, like stabby, but this will severely change
the nmide-std-lib API. But this will make Plugin development a real pain, which means I would have
to include more tests for memory leakage, functions for html-element creation/manipulation and
possibly macros.

So, I have an idea! I'll keep the same general structure on the "old" Html stuff, but use
ManuallyDrop wrapper on the fields that are used across the ABI, kids, attrs, and text. That way I
can ensure that the value that I read exists, and basically run a garbage collector every now and
then.

I need to do a deep-dive into crates like stabby and safer_ffi, to make my Rust ABI development life
easier.


\section{24.09.06}

I need a CLI-tool for creating plugins. Because I need a lot of boilerplate code for it too work,
safely and stably, in the Nmide environment. If I do not do this, I risk of creating Magnolia IDE
2, Electric Boogaloo. Which, now that I think about it, I risk either way. If I don't use crates to
make the Rust ABI stable across compiler versions, the application is kinda future safe, because for
every major/minor compiler update, you only need to recompile the IDE and Plugins you are using, and
any future compiler update will likely not break the ABI, or make it anymore unstable on the same
compiler version. While, on the other hand if I rely on crates ensuring ABI safety, a future Rust
version could break it, (insert the 1.82.0 to 1.81.0 minor update that broke a lot of crates). I
think with the crate way, I or other people could eventually fix that which is broken, but I can't
guarantee it. Working with ABI/FFI's is a pain.

I think I should work on something else for a while.


\section{24.09.08}

I have worked on something else for a while. I have found a way to have JavaScript functions, which is
cool! Now I am struggling with how to coalesce the state/html between the frontend and backend. I need
to redesign the current setup.

It sucks redesigning stuff.

\dots

I think it's best if the frontend decides when to init and update. And view should only be called after
update, but since the event system is fire-and-forget, it's so \textit{many} messages.

\begin{itemize}
  \item The frontend has an useEffect, that does the plugin script-tag creation.
  \item The frontend has an useEffect, that calls init on plugins
  \begin{itemize}
    \item It then calls the backend init, with the frontend plugin state
  \end{itemize}
  \item The backend init combines the backend plugin state with the frontend
  \begin{itemize}
    \item It then emits a "view" msg
  \end{itemize}
  \item The frontend has an useEffect that listens for "view"
  \begin{itemize}
    \item Gets the html from the frontend plugins
    \item Gets the html from the backend plugins by calling backend "view"
    \item Combines the html, and sets the react state with the result
  \end{itemize}
  \item The frontend renders the html
  \item The frontend has an useEffect, that listens for "Msg"
  \begin{itemize}
    \item It gets the state (which is stored in the backend)
    \item It calls "update" on the frontend plugins
    \item It calls "update", on the backend, with the frontend state and message
  \end{itemize}
  \item The backend "update" receives the states
  \begin{itemize}
    \item It mergers them
    \item Updates the state
    \item emits "view"
  \end{itemize}
\end{itemize}


\section{24.09.11}

The state should be stored in the frontend, and then the frontend should just call the backend.
This sucks.


\section{24.09.12}

It did not work. For some reason the application tries to allocate 93837219552704 bytes of memory,
which obviously fails. (I dont have 94\~ Terabytes of RAM, or swap, for that matter). I think I should
go back to the original plan, having the backend do the heavy lifting. State and view should be
coalesced in the backend, before the Html is sent to the frontend for rendering. But I want typing,
which the Event system in Tauri doesn't support. Or, it does, kinda. But I would like it and the IPC
to be typesafe. But oh well.

I also need to figure out a fix for pipelines.

I should really design a general structure of this IDE, and Plugins, (docs/wikis), I think it will help
to visualize stuff that way, and ensure that what I am doing is actually smart. I am beginning to worry
about delays in the IDE, especially with state and html changes. I should probably make some tests or
some concepts about this, so that I can time it. I think the general rendering of the Html will be the
most time consuming, so limiting re-rendering is of high importance. I hope I get this by using React,
with it's VDOM-system, but I should probably create something similar with the state system. If I give
the requirement for all Plugins to not have any state, (which I kinda implicitly do), I won't have to
re-do an update-call if the state is unchanged.

It's also important to find papers, articles, etc. about plugins, IDEs and similar stuff, because I think
I am just re-inventing the wheel here. I think, if I did not have the criteria of having to design
something with support for displaying Trees, I would move away from Tauri, to something more lower-level,
so that I could move away from Tauri, and hopefully use something which is easier to optimalize. But, thems
the breaks.


\section{24.09.16}

Been a few days since I've looked at this, and it doesn't suck less. I think something that would help, is
too draw the states the IDE goes through when plugins fire-off on the frontend and backend. Because right
now it's hard to visualize. I hope it will reveal some race-condition or why the application tried to
allocate so much memory. I don't think it's a huge issue, (since it probably was due to some logic flaw),
but it would be fun to figure out why.

I also think there is a need to redesign the repository, again. I think utilizing the workspace feature of
cargo will help with building, because it will then share the libraries across plugins/modules. All the C
code will not get any benefits from this.

I can't find any motivations to work on this anymore.


\section{24.09.26}

Work has been hogging a lot of my time, that and an unwillingness to work, but let's do a recap of what needs to be
done.

\begin{list}
  \item Nmide Standard Lib refactoring, check
  \item Enable support for JavaScript plugins
  \item CI Technical Debt
  \item Nmide building, check
  \item Nmide-CLI boilerplate code generation
  \item Docs
\end{list}

Missing a lot, but I think I'll be okay, as some of them are not necessary for v1.0.0 release of Nmide. Amongst them being CI,
Nmide-CLI and Docs. While nice-to-haves, they are not necessary.

I think I'll try to tackle the JavaScript issue again, by removing the support for Rust plugins, and just having states and such in
the frontend, and then the next step will be to somehow coalesce these states. It might be smarter to have the state-change-management in
the frontend, since that's where the message-listening will be.


\section{24.09.27}

I've created a Monoid. Not to flex or anything, but that means I am better than everyone else. It's for the Init function on the frontend,
which I've made, again, but this time, more functional, so it should be better. I should definitivly add comments, however, since it will
be difficult for a future student to debug this, without the same context I currently have.

I tried loading/installing plugins, this did not work. I am unsure why, but I think maybe the solution is to install them through the
backend.
