\section{24.11.02}

I have done it, I don't really know what I have done differently, but it works.
There is still the issue with Msg not being registered correctly, but I think
that's an easy fix.

We had a Magnolia Workshop this week, learned \textit{some} stuff, like, the
IDE and Compiler should communicate using LSP, (insane!). But more interesting
stuff, I have a lot of UX I need to consider, my supervisors expect IDE
functionality, and I should make a wishlist of Plugins to develop.

\begin{itemize}
  \item Plugin Debugger
  \begin{itemize}
    \item Display State
    \item Display Msg's being sent
    \item Check if there are state collisions
    \item Check if there are Msg collisions
  \end{itemize}
  \item Dependency Viewer
  \begin{itemize}
    \item Showcase JS
  \end{itemize}
  \item File Explorer
  \item Text Editor
  \begin{itemize}
    \item Use existing JS
  \end{itemize}
  \item Rendering Framework
  \begin{itemize}
    \item Facilitate ease of pre-DOM-manipulation
  \end{itemize}
  \item Mini-LSP-Client
  \begin{itemize}
    \item Hover functionality
    \item GOTO Declaration
  \end{itemize}
\end{itemize}

In state changes, plugins should only return \textit{what} they change, not the
entire state. I think this will be a good change, since not only will it ensure
that overrides are easier to detect, (since we can compare states), states from
plugins are smaller, and there will be less unnecessary duplications.

Again, unit tests should've been made ages ago, my state coalescing in the
frontend should've had unit tests. I'll make some next time.


\section{24.11.03}

What is a good way to correct state collisions? Just not allow it? Error on
collisions? Feels like it should happen at compile-time, and not runtime, but I
can't check until runtime. Maybe I should filter out plugins with collisions,
and not call them, just log it. Could be an idea. I guess that's what should be
done, since this collision only happens once, since the new architecture is to
only return the changed field, instead of the entire state. I could probably do
something like this:

// Get the state change along with the plugin name
const new_state =
  window.plugins.key_values.map(a ⇒ (fst(a), snd(b).update(msg, old_state)));
// Get duplicates
const collisions = Array.duplicates(TMap.Eq)(new_state);
// Print them
if (collisions.length !== 0) console.error("Collisions: ", collisions);
// Get unique plugins, based on plugin name
// similar to: map head \$ map group \$ map (update msg old_state) plugins, I think
setModel(A.unique(TMap.Eq)(new_state));


\section{24.11.06}

I've started work on unit tests for the frontend. I can \textit{easily} create
tests for the frontend, since all backend calls can be mocked, and I can supply
my own pseudo-react-state, which my different functions manipulate, to ensure
correctness. The plan for this month will to be finish the wanted
functionality for v1.0.0, two months late. I just need to figure out some good
tests.

I've spent like 5 hours, (with breaks), writing tests just for my utility
functions. This is going to take weeks.


\section{24.11.07}

Discovered something interesting. I can basically mock the entirety of the
frontend. So, testing JSPS can be really extensive, ensuring the logic holds.
I am knee-deep in \dots I don't know, Functional Programming? Logic?
Category Theory? Creating equalities, monoids and such for my types, so that I
can use the fp-ts library fully. I have the Haskell implementation of groupBy
open so that I have a starting of point for it in my own code.

I got to the point of writing tests for state-updating, and then I figured I
should rewrite my update function, to add logging for state-collisions. Similar
to what I discussed the other day. I think it'll work great.


\section{24.11.08}

Plugins should be a list of Plugin Name, and the Plugin, so that one, (me), can
narrow down which Plugin causes what problem.

It is super boring to write tests. This is killing me. Wrote a fun function
for state change, which made me feel smart, but I also forgot the name of
\texit{unzip}, which made me feel stupid. I was talking to some other students,
and they pointed out something I should probably mention. I am writing
functional TypeScript. Which is not a thing, in the sense it is not optimized.
So my code might be really slow. Hopefully this is not the case. Either way,
this is something I can see easily with E2E-testing, which I only have for the
frontend, not the backend. I think I'll aim for writing these test's til the end
of this month, and then start on creating the dependency plugin, in TypeScript.
Hopefully I've found a smart solution for compiling TypeScript into JavaScript.
And it will be a good test of the plugin system.

I think I am going to say I am done with unit tests for today. I also did some
logic-changes, (which I discovered when creating the tests), so I want to see
if I can get the counter example to work. Then maybe I'll look into some
JSP-Util library.

Took one minute. I think I am starting to get a good system/architecture.

It is pretty simple to create a JSP-Util Library, the only issue is that LSP
help can be tricky to do. But that is a future problem, if it even is a
problem.
<<<<<<< HEAD


\section{24.11.09}

I think I can release v1.0.0, just need to write some documentation, and then
I'll build some files, and finish.

I can build for Windows and Linux, but building for macOS means I have to get
my hands on a macOS machine or a WM, or get my CI/CD to work again.

Tried to do CI/CD stuff, but GitLab-runner issue is a recurring problem. The
ones I tried to create have not been online for two months, (about the time
since I did CI/CD work), and the universities' is also offline! I hope it comes
online tomorrow, and that I remember to turn on my runner.


\section{24.11.10}

Back on the CI/CD-grind, this sucks. After I migrated to Tauri-2.0, and
migrated to ViteJS, my Tauri-Image is missing dependencies, and it's the same
process as before, building the image, (which takes 11 minutes, I should've
fixed this a long time ago), and then re-running the pipeline. Just to find out
I am missing another library. I feel Cargo should know what dependencies crates
have, but this is only discovered during compile-time.

It takes around 16 minutes to build now, I hate doing CI/CD. But I am close to
getting a NSIS build working, i.e. Windows build to work without it being
Windows.


\section{24.11.09}

I think I can release v1.0.0, just need to write some documentation, and then
I'll build some files, and finish.

I can build for Windows and Linux, but building for macOS means I have to get
my hands on a macOS machine or a WM, or get my CI/CD to work again.

Tried to do CI/CD stuff, but GitLab-runner issue is a recurring problem. The
ones I tried to create have not been online for two months, (about the time
since I did CI/CD work), and the universities' is also offline! I hope it comes
online tomorrow, and that I remember to turn on my runner.


\section{24.11.14}

Started utilizing the issue system in GitLab, my thought process is that this
will make it easier to both keep track on what I am working on, and make it
easier to switch between tasks, because some of them are less interesting.


\section{24.11.15}

The only problem with doing this issue-system, is that taking notes is a little
bit harder.

I have made plugin installation the way I wanted, as script-tags, and made a
demonstration of how to use this system alongside an external JavaScript
library. But I've done some thinking, do I need to use React? I don't think so.
But, it does give a lot of handy things for free. Like optimized rendering,
\textit{easy} update handling.
I think I should at least analyze it, and write a few paragraphs about my tech
stack choice, among them being React.

Maybe Haskell?


\section{24.11.16}

I think it would be possible to do what Tauri does, but in Haskell, with
something like threepeny-gui https://heinrichapfelmus.github.io/threepenny-gui/
and electron https://www.electronjs.org/. This would be cool, but I think I
should focus on creating the first few plugins, and maybe look at re-writing
everything to Haskell. I think it would be a fun experience.

But, to the real matter at hand. My current Plugin setup, should work
regardless of what tech stack I have. I have a trivial-JSP, that renders the
dependency graph using the D3 JS-library. It does this by exposing an extra
function, that another JSP is calling, when it receives a Msg.

All I need now, are two Plugins, one that acts as a file explorer, and another
that can parse the dependencies between the files. Maybe a mini-LSP?

Something has happened, so Rust plugins don't work anymore. I don't know what,
but I realized something, I would like to debug Rust Plugins, like I debug
JSP's, being able to interact with them. There are a few ways I can achieve
this. I could add this functionality to the CLI tool I want to make, or I
could change the way I do backend plugins, install them the same way I install
JSP. This would just be an invoke-call.


\section{24.11.17}

I am going to redesign the standard library to fix memory leaks.

I think I should refactor the Tauri app to just Purescript, I think just
utilizing the "purity" of plugins should be enough when re-rendering.
Furthermore, I hope I can find something that does the V-DOM stuff for me. If
just the window object is used, it would be easy to just inject functions
wherever, like the state validation logic. Or rendering. We'll see, for now I'll
just "half-ass" this dependency viewer plugin.

I think the dependency viewer thing works, but Magnolia is so large, it takes a
long time to parse it into a tree. I need to create some efficient way to
navigate it.


\section{24.11.22}

Done some plugin work, the dependency viewer works, it was not Magnolia, but
skill-issue that was the issue. I tried to read the entire .svn folder, which
I assume is a large one. Did some change on how I do stuff as well, instead of
doing stupid splitting of the file, I changed it to regex. The graph looks
ugly, but it works. If I want to make it prettier, I need to be better at D3,
the JavaScript visualization library I am using. Which I cannot be bothered to
do. Instead, I will create the debug-plugin, which should let me enable and
disable plugins. I kinda want to do this in PureScript, but it is frustrating 
to work with, since Array and lists are two different things, and I would
rather be working with Haskell, but I cannot be bothered with learning to
compile it to JavaScript.

The more I am working with PureScript, the more annoyed I am becoming. Why are
same concepts in PureScript not translated as the same concept in JavaScript?
It not being a one-to-one mapping on basic stuff like tuples is very annoying.

The encoding/decoding is so close, and to \textit{fix} it I have to write so
much shitty code.

Going to TypeScript did not help either, because, yet again I am hindered by
my shitty code. Not having tests is a pain, I have written about this before,
and what did I learn? Apparently nothing!
Still got state issues. I need to expand the current tests to include the
plugins I have made, to ensure a wider set of test data.


\section{24.11.23}

I just forgot to use my model-fold function. The whole idea about the new
re-write, (having update return the updated fields instead of the entire
state), was to avoid needlessly sending the state around. But for that to work,
one needs to have the previous state, and overwrite the fields in it, with the
new values from the coalesced update.

Now, for my debug plugin to work, I need to overwrite all plugins. Which I
should be able to do quite effectively. I can just map some function over the
plugin-installation-field, and return the empty result if it is disabled. This
can also be done in the init function, since it is by default enabled, so it
does not matter which order this pre-write happens.

I don't think React is doing anything to help, since the re-rendering of the
entire DOM happens every time the state changes. I think this is because I do
not use the \textit{key} attribute. Re-doing and re-thinking the way things are
rendered will help, because the only reason React was used, was for this case,
maybe going to another framework, or no framework, would be a good choice.
Because now, the only reason to keep React, is the way it handles
React-State-Change with regard to perpetually re-calling the listeners, which
most likely can be emulated. I will look into it.


\section{24.11.24}

I think the re-write of the project should be to TypeScript first, and then
PureScript. There aren't a lot of examples of Tauri + PureScript inter-op out
there so, from what I can tell, I still need at least an entry-point, that calls
the main function to the PureScript code. So instead of going to the extreme
right away, I'll do it partially. That way I can also help create more examples
of PureScript + Tauri, since I have already made the only example app for Tauri
+ PureScript.

So this works, but I need to handle how to re-render stuff. I want plugins to
be able to optionally clean up after they have done a view. So in most cases,
the view can just be called once, and every subsequent call, it only changes what
html it wants to change. This off course only works in the JSPS, all other
plugins without direct access to the DOM need to re-render everything. My idea,
is that the Core can sort the HTMLElements created by which plugin created them,
and then remove them when needed. The problem is that I am now moving away from
the pure Init-Update-View-architecture I designed, so I should probably wrap
this idea into its own JS-Plugin.


\section{27.11.24}

I still have the Init-Update-View-architecture, (IUV), all I am doing is
exposing more of the internal API to JSPS. I think this is okay, since
theoretically, nothing is different between the \textit{standard} plugin system,
(SPS), and the JSPS, but practically, a JSP would be more suited for managing
the user view, while the SPS would be better suited for more computational heavy
tasks. Maybe, I have no source to back me up here, just vibes. I still have the
recurring problem of state management. I am hoping this will be solved when I
rewrite my IUV-System to PureScript. The issue will be that the ADT's in
PureScript do not map to JavaScript Objects. Sometimes they are wrapped, which
makes the FFI between PureScript and JavaScript difficult. But I think just
converting the objects before they enter and after they leave the PureScript
world will suffice. I was also thinking, that since everything is managed on
the frontend side, I should probably move away from the T-types, (THtml, TMap,
etc.), I think converting them to the actual JavaScript Objects they represent
will be better. I would just have to keep the utility functions I have made for
managing the types in the frontend, and also add them to the backend.

Plugins should be able to have different displays. Or should that be a Plugin?
I am developing a thing for plugin-izing, all possible features. If it is a
Plugin, then I need to expose even more of the API. I have to change the way
Plugins are called, by adding an extra key, being the window-url.

Found out that the way I do collisions, don't really make sense. I had this
typing: (TMap, [(String, TMap)]), Where the first element of the tuple, is the
final new state, while the second, is a list of collisions, the plugin and the
field that is collided with. This information is not encoded in this type.


\section{28.11.24}

I don't think I'll be able to get good PureScript support by the end of this
month, but this is okay, since it was not the goal of this month. The general
plan for the coming months should be developing the actual IDE plugins. I think
I should see if I can make stuff abstract first, like how the setup should be
for the plugins. It is probably smart to create the \textit{graph} of how the
IDE plugins. Like, this plugin does this stuff, which this plugin can use.

