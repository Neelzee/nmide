\section{25.01.04}

Good start of the new year, working on a rewrite of my project. I've run into
some possible issues, only issues because I would like stuff to be simple for
module developers, but I think I'll start by doing it the \textit{naive} way,
which is easier for me, as a maintainer/core-developer to do now, and maybe
change in the future. I don't think any of these possible future changes will be
as major as this, anyway. The issue with module loading, where I had a timeout
to wait until all modules where loaded, by just waiting 250 ms, is solved, in a
smart way, I think, atleast. Everytime an installer \textit{installs} a module,
it increments a global counter. Everytime a module is loaded, (pushed to a map),
the same global counter is decremented. Finally, every 50 ms the core checks if
the counter is 0. The only way for it to be less than 0, is if a single script,
\textit{installs}/\textit{loads} more than one module, which can be noted in the
wiki as \bold{DONT}. So that is a module developer skill issue.

Thought I was close to finished with the rework, but I forgot to make my
instruction set, (the tree-DSL), JSON-safe, (something that can be represented
using just JSON). I have to rework that system, because originally, it was just
two instructions, removing a node, and modifying a node. Removing a node just
needs the id of the node to be removed, while modifying the node, had a function
which is applied to the original node. I had something earlier, where the modify
instruction had the field and value of the node, but this did not work, or I
gave that up, since I needed/wanted the tree-system to be generic, and for that,
the user needs to supply their own function, which, offcourse cannot be done
with JSON. The reason I did this, was because I could not find an easy way to
type the modify instruction, such that the value and field that the module
supplied is constrained to a valid type on the node. For example, if I want to
modify the state, on the field "errors", which is a list, I could only supply a
new array, not just the element I want to append, or if I wanted to filter it, I
could not do that either. It is possible to just have it like that, where the
module supplies the new value, and handles the contamination on their own, but
how can I do that, while keeping the instruction system? Maybe I could let the
modules directly manipulate the nodes, and just diffing the nodes? That's the
same problem to the tree one, just on the leaves instead. The main issue, I
think is the attributes field on THtml, because that is what I see as something
a module wanting to change the most.
Now that I am thinking about it, I don't think I've really thought out how
communication between the Core and modules should be. And especially not about
how a future backend is supposed to supply their own modules.

Or do they? Say hello to my new favourite module:
\textit{all_i_do_is_call_the_backend.js}. Then stuff can be the same, and the
supplied function to the modify instruction is \textit{JSON-safe}. But not
really, because we don't care about that. Just have to change the modify
instruction to return a promise instead. Or the Task type.

\section{25.01.08}

I have to rework my rework. Kinda, I've forgotten what I was working on, what's
missing and such.

This sucks. The complexity of the \textit{everything-is-a-tree-system} keeps
bubbling up to the module developer. I think I have to have one instruction set
for each field, instead of the generic one I have, so that I can \textit{easily}
coalece the ui, state and events after a core modification, and as an MVP, I'll
just pretend everything is always good, and just drop all modifications that
gives any issues.

\section{25.01.09}

I have to timebox the rework. I am breaking my brain trying to do it
\textit{functionally} with fp-ts, in Typescript. Everything is overly
complicated and confusing. If I don't have anything by the end of the 18th, I'll
\sout{kms} remove fp-ts from dependencies, (move verification of input to own
library), and write the Core using just pure Typescript, no libraries. Which
will suck, and not be fun, but I can't spend all my time doing this useless
work, and I can't get my setup to work with purescript, or fay, so I am stuck
with Typescript until I hand this in. I have definitively prioritized my time
wrong. Should've spent more time actually \textit{developing} modules, and not
adding support for other languages. No one cares about that.

\section{25.01.24}

Long time since I've written anything, little less time since I've been doing
anything. Nothing works. I need to find a better way to do what I am trying to
do, so that stuff works, but it's been a while so I don't really know what I
wanted to do, or what I am doing.

Rendering does not work, I don't know how to fix it.


\section{25.01.29}

I have to find something to do. I am feeling burned out. I don't want to work
anymore. I guess I could go back to working on my proto- thesis and presentation,
that is due to march. Maybe I'll get some inspiration to work on code again.


\section{25.01.30}

I started a little bit on the shortform/proto- thesis and presentation during
the christmas holiday. This is was the general layout for the presentation:

\begin{itemize}
  \item What an IDE is
  \item Why a create a new IDE
  \item Features
  \item Modularization
  \item Techstack
  \item Module Families and Granularity
  \item Conclusion
\end{itemize}

Which I want to expand/revise to this:

\begin{itemize}
  \item Creating an IDE
  \begin{itemize}
    \item Why create a new IDE
    \item Why modular
  \end{itemize}
  \item Features
  \begin{itemize}
    \item Modularization
    \item Tech stack
  \end{itemize}
  \item Challenges
  \begin{itemize}
    \item Refactoring
    \begin{itemize}
      \item Granularity
      \item Module Families
    \end{itemize}
  \end{itemize}
  \item Conclusion
\end{itemize}

I'll timebox this work til the end of Februrary, only because the shorterm-stuff
is due to 4th of March, so it is timeboxed. I kinda wanna learn nix so that I
can build my pdf and slides with it, instead of having to memorize a latex build
command. I think that will be a nice \textit{break} from doing actual work.

\section{25.01.31}

I haven't really done much, except the nix stuff. Which has been fun, and I can
now build both my slides and thesis, reproducibly. Now I just have to actually
write stuff. Much of what I wrote during the holiday break, has to be
re-written, which is not surprising, but it is demotivating. Most of it was bad
regardless, the structure and all that.
