\chapter{Optimizing instructions sets} \label{app:a}

The following appendix discusses how we can optimize an instruction set. When we
use the term \textit{optimize}, we mean to reduce the amount of instructions.
This can be achieved in a few different steps. \textbf{Instruction} is a
recursive data type, parameterized by \textbf{T}. Since an \textbf{Instruction}
form a group\footnote{See section \ref{sec:moD3} in chapter \ref{cha:impl}}
structure, we know for any \textbf{Instruction} variant, there exist an inverse,
giving us the unit. The goal of this optimization, is to see if an
\textbf{Instruction} tree contains any inverses.

\begin{enumerate}
  \item Flatten the Instruction
  \item Register how many times a field is modified
    \begin{enumerate}
      \item Turn unnecessary Instructions into NoOps
    \end{enumerate}
  \item Filter out NoOps
  \item Unflatten the Instructions
\end{enumerate}

An \textit{unnecessary} operation is one that leads to an \textbf{NoOp}, which
is the case of inversable \textbf{Instruction}s. The optimalization function
is not parameterized by a strict variant of \textbf{T}. In the \gls*{ide} the
\textbf{Instruction}s are parameterized by \textbf{Value}, \textbf{Html},
\textbf{Attr} and \textbf{String}. The last three are specific to \gls*{ui}
modification, and the last two are modifications on specific \textbf{Html}
instances. This means that if an \textbf{Instruction} parameterized by
\textbf{Html}, is of the \textbf{Rem} variant, we know all other \gls*{ui}
\textbf{Instruction}s pertaining to the removed \textbf{Html} variant are
\textbf{NoOp}s.

\begin{code}[H]
  \lstinputlisting
   [ language=Rust
   , caption={Instruction (Rust)}
   , label=lst:aInst
   , firstline=4
   , lastline=16
   ]{./libs/rust/core-std-lib/src/instruction/inst.rs}
\end{code}

In Rust we can have generic data types, as shown in listing \ref{lst:aInst}, by
the type parameter, $T$, but we have to restrict the type $T$ to a type that
implements the trait \textbf{PartialEq}, which means we can use equality on it.
We need this restrictions, because the attribute macro \textbf{Instruction} has.
These macros generate the needed code to implement the different traits:

\begin{itemize}
  \item Debug: Enables the implementer to be printed to \textit{stdout}
  \item Default: Implements a default variant of the implementer type, in this
    case, NoOp
  \item Clone: Implements a simple \textbf{clone} method, to create an owned
    instance of a borrowed value
  \item Deserialize \& Serialize: Implements the needed methods for encoding
    and decoding a variant to a \gls{json} representation
  \item TS: Enables automatic TypeScript type generation of the variant
\end{itemize}

We first start the optimalization step, by removing all NoOps, and then
flattening the instructions, by using the \textbf{opt} and \textbf{flatten}
methods, shown in listings \ref{lst:opt} and \ref{lst:flatten} respectively.

\begin{code}[H]
  \lstinputlisting
   [ language=Rust
   , caption={
     Opt method (Rust): Uses a match statement and a guard to match on a
     \textit{slice}, (reference to a Vec). The guard lets us add a predicate to
     our branch, in this case, if y \textit{matches} an NoOp. If it is an empty
     slice, it's a NoOp, otherwise, it will be an Instruction with all NoOps
     recursively removed.
   }
   , label=lst:opt
   , firstline=119
   , lastline=151
   ]{./libs/rust/core-std-lib/src/instruction/opt.rs}
\end{code}

\begin{code}[H]
  \lstinputlisting
   [ language=Rust
   , caption={
     Flatten method (Rust): Note the lack of return statements, this is because
     the last expression in a function in Rust, is returned, if the expression
     does not end with a semicolon.
   }
   , label=lst:flatten
   , firstline=17
   , lastline=26
   ]{./libs/rust/core-std-lib/src/instruction/opt.rs}
\end{code}

In listing \ref{lst:count}, we then iterate over each instruction in the
sequence, and mapping each field and value to a counter. If it's an Add
instruction, the counter is incremented, if it's a Rem instruction, the counter
is decremented. We don't have a way to inform the compiler that we have removed
all NoOp and Then instructions, and we need complete match-statements, so we add
a catch-all with an \textit{unreachable} macro, which will \textit{panic} with
the supplied message. This is commonly used to represent a state that is
unreachable, but something the compiler can't prove.

\begin{code}[H]
  \lstinputlisting
   [ language=Rust
   , caption={Modification counting (Rust)}
   , label=lst:count
   , firstline=50
   , lastline=79
   ]{./libs/rust/core-std-lib/src/instruction/opt.rs}
\end{code}

Finally, in the listing \ref{lst:fold}, we \textit{unflatten} the sequence of
instructions, and check the count for each Add and Rem Instruction. If it is
above $0$, then that means we have added that field-value pair more times than
removing it, but we can still only add it once, so we set the count to $0$, and
return a Then instruction, since we have the accumulated instructions along with
the current Add instruction. If the count is less than $0$, then it means we are
removing it more times than adding it, similarly, we can only remove it once, so
we set the count to $0$, and combine the accumulated instruction, with the Rem
instruction. Because of our \textbf{combine} implementation, we can be sure that
the initial NoOp element is removed as soon as possible.

\begin{code}[H]
  \lstinputlisting
   [ language=Rust
   , caption={Instruction folding (Rust)}
   , label=lst:fold
   , firstline=81
   , lastline=112
   ]{./libs/rust/core-std-lib/src/instruction/opt.rs}
\end{code}
