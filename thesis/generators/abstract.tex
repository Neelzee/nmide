\pagenumbering{roman}

\begin{abstract}

  This paper introduces a Modular, \textit{Zero Core}, Application, to serve as
  an \gls{ide} for experimental programming languages, addressing limitations in
  traditional \gls{ide}s. While standard \gls{ide}s are
  crucial in software development, their support for experimental languages is
  often inadequate. This can and, has been solved by extensively using the module
  architecture of existing \gls{ide}s. However, relying on \textit{niche} modules
  or functionality is beneficial for the longevity of the software.
  This project proposes a modular \gls{ide} to extend its lifespan and enhance
  support for experimental languages. Analyzing the essential features of
  traditional \gls{ide}s and the need for adaptability to new paradigms and
  tools. Magnolia, a research programming language developed at the University
  of Bergen, serves as a case study, highlighting its unique characteristics and
  the necessity for a modular \gls{ide}. The primary research question explores how
  modularization facilitates the design and implementation of experimental
  programming languages. To showcase the usefulness of a modular approach, the
  modules needed to extend the core application to an \gls{ide} will be
  implemented.
  \keywords{Modularization \and \gls{ide} \and Magnolia.}
\end{abstract}

\renewcommand{\abstractname}{Acknowledgements}
\begin{abstract}
  Est suavitate gubergren referrentur an, ex mea dolor eloquentiam, novum ludus
  suscipit in nec. Ea mea essent prompta constituam, has ut novum prodesset
  vulputate. Ad noster electram pri, nec sint accusamus dissentias at. Est ad
  laoreet fierent invidunt, ut per assueverit conclusionemque. An electram
  efficiendi mea.

  \vspace{1cm}
  \hspace*{\fill}\texttt{Nils Michael Fitjar}\\
  \hspace*{\fill}\today
\end{abstract}
\setcounter{page}{1}
\newpage
