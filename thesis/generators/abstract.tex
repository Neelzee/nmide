\pagenumbering{roman}

\begin{abstract}
  This paper introduces a modular, \textit{zero-core}, application, to serve as
  an \gls*{ide} for experimental programming languages, addressing limitations
  in traditional \gls*{ide}s. While standard \gls*{ide}s are crucial in software
  development, their support for experimental languages is often inadequate.
  This can be mitigated by extensively using the module
  architecture of existing \gls*{ide}s, by creating specific modules to address
  the shortfall of the host \gls*{ide}. However, relying on \textit{niche}
  modules or functionality is not beneficial for the longevity of the software.
  By analyzing the essential features of traditional \gls*{ide}s a need for
  adaptability by \gls*{ide}s to new paradigms and tools is highlighted. The
  solution, proposed by this paper, is to utilize a modular architecture to
  extend its lifespan and enhance support for experimental languages.
  Magnolia, a research programming language developed at the University of
  Bergen, serves as a case study, highlighting its unique characteristics and
  the necessity for a modular \gls*{ide}. The primary research question explores
  how modularization facilitates the design and implementation of experimental
  programming languages. To showcase the usefulness of a modular approach, the
  modules needed to extend the core application to an \gls*{ide} will be
  implemented.

  \keywords{Modularization \and \gls*{ide} \and Magnolia.}
\end{abstract}

\renewcommand{\abstractname}{Acknowledgements}
\begin{abstract}
  I would like to thank my supervisors, Magne Haveraaen and Mikhail Barash for
  their valuable feedback and discussions. Finally, I would also like to thank
  my friends and family, for their support throughout my studies.

  \vspace{1cm}
  \hspace*{\fill}\texttt{Nils Michael Fitjar}\\
  \hspace*{\fill}\today
\end{abstract}
\setcounter{page}{1}
\newpage
