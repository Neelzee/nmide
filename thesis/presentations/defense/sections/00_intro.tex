\section{Introduction}
\SectionPage

\begin{frame}
  \frametitle{Topic}
  \begin{itemize}
    \item About my thesis
      \pause
    \item What an IDE is
      \pause
    \item About Magnolia
      \pause
    \item My implementation
      \pause
    \item Results
      \pause
    \item Demo
  \end{itemize}
\end{frame}

\begin{frame}
  \frametitle{Thesis}
  \begin{itemize}
      \pause
    \item For the technically inclined:
      \begin{itemize}
        \item Design and develop a \textit{zero-core}, \textit{modular}, \textit{IDE} 
      \end{itemize}
      \pause
    \item For family members:
      \pause
      \begin{itemize}
        \item Creating a tool for developers
      \end{itemize}
      \pause
    \item Modular
      \pause
      \begin{itemize}
        \item Functionality can be added by other developers (modules)
      \end{itemize}
      \pause
    \item Zero-core
      \pause
      \begin{itemize}
        \item The tool gets all of its functionality from modules
      \end{itemize}
  \end{itemize}
\end{frame}

\begin{frame}
  \frametitle{Integrated Development Environment (IDE)}
  \begin{itemize}
      \pause
    \item All-in-one tool
      \pause
    \item Single application with \textit{everything} a developer needs
      \pause
      \begin{itemize}
        \item Manage files
        \pause
        \item Compile and run code
        \pause
        \item Run tests and show test reports
        \pause
        \item Check for errors
        \pause
        \item Warn about possible bugs
        \pause
        \item Grammar check
      \end{itemize}
    \pause
    \item Common features across languages and IDEs
    \pause
    \item Might be unsuited for languages with unconventional features
  \end{itemize}
\end{frame}

\section{Magnolia}
\SectionPage

\begin{frame}
  \frametitle{Bergen Language Design Laboratory}
  \begin{itemize}
    \item Researches at BLDL are experimenting with a programming language
    called Magnolia
      \pause
    \item Created to experiment with novel language features inspired by
    academia
      \pause
      \begin{itemize}
        \item For example: abstract algebra
      \end{itemize}
      \pause
    \item Which may be out-of-scope for IDEs
  \end{itemize}
\end{frame}

\begin{frame}
  \frametitle{Magnolia}
  \begin{itemize}
    \item Introduces something called \textit{concepts}
      \pause
    \item Similar to a Java interface.
      \pause
    \item A concept declares
      \begin{itemize}
      \pause
        \item Types
      \pause
        \item Operations on those types
      \pause
        \item Axioms that specify the properties of the operations
      \end{itemize}
      \pause
    \item A concept can use other concepts, and rename the types and operations
      in the concept, this is called renaming
      \pause
    \item Renaming is useful to vizualise for a Magnolia developer, but not a
    common feature in IDEs
  \end{itemize}
\end{frame}

\begin{frame}
  \frametitle{Group is a magma with more properties}
  \begin{itemize}
    \item Magma is the \textit{simplest} structure in abstract algebra
      \pause
    \item It is trivial to specify in Java using interfaces
      \pause
    \item Group, which is an exstension of magma, is more complicated
      \pause
    \item Group \textit{extends} monoid, \textit{extends} semigroup,
      \textit{extends} magma
      \pause
    \item With \textit{inverse element}, \textit{identity}, and
      \textit{associativity} properties (\textit{interfaces}) respectively
  \end{itemize}
\end{frame}

\begin{frame}
  \frametitle{Magma example}
  \begin{figure}[H]
    \begin{subfigure}[h]{0.45\textwidth}
      \begin{equation}
        M = \{ a, b, c, \dots \}
      \end{equation}
      \begin{equation}
        (M, \bullet)
      \end{equation}
      \begin{equation}
        a \bullet b \in M
      \end{equation}
    \end{subfigure}
    \hfill
    \begin{subfigure}[h]{0.45\textwidth}
      \begin{center}
        \pause
      \lstinputlisting
      [ language=Java
      ]{./code/magma.java}
  \end{center}
    \end{subfigure}
  \end{figure}
\end{frame}

\hidelogo

\begin{frame}
  \frametitle{From semigroup to group}
  \begin{figure}[H]
    \begin{subfigure}[h]{0.45\textwidth}
      Associativity (Semigroup)
      \begin{equation}
        (a \bullet b) \bullet c = a \bullet (b \bullet c)
      \end{equation}
    \end{subfigure}
    \hfill
    \begin{subfigure}[h]{0.45\textwidth}
      \pause
  \begin{center}
    \lstinputlisting
    [ language=Java
    ]{./code/semigroup.java}
  \end{center}
\end{subfigure}
  \end{figure}
      \pause
      \begin{equation}
        (1 + 2) + 3 = 1 + (2 + 3)
      \end{equation}
      \pause
  \begin{figure}[H]
    \begin{subfigure}[h]{0.45\textwidth}
      Identity (Monoid)
      \begin{equation}
        a \bullet e = a
      \end{equation}
    \end{subfigure}
    \hfill
    \begin{subfigure}[h]{0.45\textwidth}
      \pause
  \begin{center}
    \lstinputlisting
    [ language=Java
    ]{./code/monoid.java}
  \end{center}
\end{subfigure}
  \end{figure}
      \pause
      \begin{equation}
        2 + 0 = 2
      \end{equation}
      \pause
  \begin{figure}[H]
    \begin{subfigure}[h]{0.45\textwidth}
      Invertability (Group)
      \begin{equation}
        a \bullet b = e
      \end{equation}
    \end{subfigure}
    \hfill
    \begin{subfigure}[h]{0.45\textwidth}
      \pause
  \begin{center}
    \lstinputlisting
    [ language=Java
    ]{./code/group.java}
  \end{center}
\end{subfigure}
  \end{figure}
      \pause
      \begin{equation}
        10 + (-10) = 0
      \end{equation}
\end{frame}

\showlogo

\begin{frame}
  \frametitle{Magma example in Magnolia}
  \begin{center}
    \lstinputlisting
    [ language=Magnolia
    ]{./code/magma.mg}
  \end{center}
\end{frame}

\begin{frame}
  \frametitle{Semigroup example in Magnolia}
  \begin{center}
    \lstinputlisting
    [ language=Magnolia
    ]{./code/semigroup.mg}
  \end{center}
\end{frame}

\begin{frame}
  \frametitle{The current Magnolia IDE}
  \pause
  \begin{itemize}
    \item Directly integrated with the Magnolia Compiler
      \pause
    \item Made using a (now) old version of Eclipse
      \pause
    \item Uses (now) deprecated Eclipse plugins
      \pause
    \item Installation process is complex
      \pause
    \item In INF220, two weeks is set aside for students to install it
      \pause
    \item But has functionality to show renaming
  \end{itemize}
\end{frame}
