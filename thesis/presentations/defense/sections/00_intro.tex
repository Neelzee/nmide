\section{Introduction}
\SectionPage

\begin{frame}
  \frametitle{Topic}
  \begin{itemize}
    \item About my thesis
    \item Modularity
    \item Implementation
    \begin{itemize}
      \item Conclusion
      \item Demo
      \item Modules
    \end{itemize}
  \end{itemize}
\end{frame}

\begin{frame}
  \frametitle{Thesis}
  \begin{itemize}
    \item For family members:
      \begin{itemize}
        \item Creating a tool for developers
      \end{itemize}
    \item For the technically inclined:
    \begin{itemize}
      \item Designing and developing a \textit{zero-core}, \textit{modular}, \textit{IDE}
    \end{itemize}
    \item Zero-core
      \begin{itemize}
        \item The tool gets all of its functionality from modules
      \end{itemize}
    \item Modular
      \begin{itemize}
        \item Functionality can be added by other developers (modules)
      \end{itemize}
  \end{itemize}
\end{frame}

\begin{frame}
  \frametitle{Integrated Development Environment (IDE)}
  \begin{itemize}
    \item All-in-one tool
    \item Single application with \textit{everything} a developer needs
      \begin{itemize}
        \item Manage files
        \item Compile and run code
        \item Run tests and show test reports
        \item Check for errors
        \item Warn about possible bugs
        \item Grammar check
      \end{itemize}
    \item Common features across languages and IDEs
    \item Might be unsuited for languages with unconventional features
    \item Can be solved with a module architecture
  \end{itemize}
\end{frame}

\section{Modularity}
\SectionPage

\begin{frame}
  \frametitle{Modularity in a programming language}
  \begin{itemize}
    \item Functions, libraries, and frameworks
    \item Researches at UiB are experimenting with a programming language
    called Magnolia
    \item Created to experiment with novel language features inspired by
    academia
      \begin{itemize}
        \item For example: abstract algebra
      \end{itemize}
  \end{itemize}
\end{frame}

\begin{frame}
  \frametitle{Modularity in mathematics}
  \begin{itemize}
    \item Structure: a set, with an operation, and some \textit{rules} on the
    operation
    \item Magma is the \textit{simplest} structure in abstract algebra
    \item It is trivial to specify in Java using interfaces
    \item Semigroup, which is an exstension of magma, is more complicated
    \item With \textit{associativity} property (\textit{interface})
    \item More \textit{properties}, more interfaces
  \end{itemize}
\end{frame}

\begin{frame}
  \frametitle{Magma}
  \begin{figure}[H]
    \begin{subfigure}[h]{0.45\textwidth}
      \begin{equation}
        M = \{ a, b, c, \dots \}
      \end{equation}
      \begin{equation}
        a \odot b \in M
      \end{equation}
    \end{subfigure}
    \hfill
    \begin{subfigure}[h]{0.45\textwidth}
      \begin{center}
      \lstinputlisting
      [ language=Java
      ]{./code/magma.java}
  \end{center}
    \end{subfigure}
  \end{figure}
\end{frame}

\hidelogo

\begin{frame}
  \frametitle{Semigroup}
  \begin{figure}[H]
    \begin{subfigure}[h]{0.45\textwidth}
      Associativity (Semigroup)
      \begin{equation}
        (a \odot b) \odot c = a \odot (b \odot c)
      \end{equation}
    \end{subfigure}
    \hfill
    \begin{subfigure}[h]{0.45\textwidth}
      \begin{center}
        \lstinputlisting
        [ language=Java
        ]{./code/semigroup.java}
      \end{center}
    \end{subfigure}
  \end{figure}
  Addition with the natural numbers is a semigroup\footnote{Its technically a monoid, but a monoid is a semigroup with the identity property}
      \begin{center}
        $(1 + 2) + 3 = 1 + (2 + 3)$
      \end{center}
\end{frame}

\begin{frame}
  \frametitle{Magnolia: modularity in a programming language}
  \begin{itemize}
    \item Introduces something called \textit{concepts}
    \item Similar to a Java interface.
    \item A concept declares
      \begin{itemize}
        \item Types
        \item Operations on those types
        \item Axioms that specify the properties of the operations
      \end{itemize}
    \item Allows for reuse of logic, not just functions
    \item A concept can use other concepts, and rename the types and operations
      in the concept, this is called renaming
    \item Renaming is useful to vizualise for a Magnolia developer, but not a
    common feature in IDEs
  \end{itemize}
\end{frame}

\showlogo

\begin{frame}
  \frametitle{Magma example in Magnolia}
  \begin{figure}[H]
    \begin{subfigure}[h]{0.45\textwidth}
  \begin{center}
    \lstinputlisting
    [ language=Magnolia
    ]{./code/magma.mg}
  \end{center}
    \end{subfigure}
    \hfill
    \begin{subfigure}[h]{0.45\textwidth}
      \begin{center}
        $M = \{ a, b, c, \dots \}$
      \end{center}
      \begin{center}
        $a \odot b \in M$
      \end{center}
    \end{subfigure}
  \end{figure}
\end{frame}

\begin{frame}
  \frametitle{Semigroup example in Magnolia}
  \begin{figure}[H]
    \begin{subfigure}[h]{0.45\textwidth}
  \begin{center}
    \lstinputlisting
    [ language=Magnolia
    ]{./code/semigroup.mg}
  \end{center}
    \end{subfigure}
    \hfill
    \begin{subfigure}[h]{0.45\textwidth}
      Associativity (Semigroup)
      \begin{center}
        $(a \odot b) \odot c = a \odot (b \odot c)$
      \end{center}
    \end{subfigure}
  \end{figure}
\end{frame}

\begin{frame}
  \frametitle{Modularity in an IDE}
  \begin{itemize}
    \item IDEs functionality can be extended with modules
    \item Third party code to be executed/interpreted
      \begin{itemize}
        \item Tailor made Scripting Language
          \begin{itemize}
            \item Vim Script (Vim)
          \end{itemize}
        \item An already existing programming language
          \begin{itemize}
            \item Lua (NeoVim)
            \item JavaScript (VS Code)
            \item Java/Kotlin (Eclipse/IntelliJ)
          \end{itemize}
      \end{itemize}
  \end{itemize}
\end{frame}

\begin{frame}
  \frametitle{The current Magnolia IDE}
  \begin{itemize}
    \item Directly integrated with the Magnolia Compiler
    \item Made using an (now) old version of Eclipse
    \item Uses (now) deprecated Eclipse modules
    \item Installation process is complex
    \item In INF220, two weeks is set aside for students to install it
    \item But has functionality to show renaming
  \end{itemize}
\end{frame}
