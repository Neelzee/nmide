\section{Introduction}
\SectionPage

% NOTE: This is a vague simili on software development and building a house,
% specifically pointing out how _rigorus_ development and builiding is, as most
% developers do not take into account of changing priorities when developing.
\begin{frame}
  \frametitle{Software development}
  \begin{itemize}
    \item Like building a house
    \pause
    \begin{itemize}
      \item Strong foundation
      \pause
      \item Built for a purpose
      \pause
      \item Renovations/remodeling is expensive
      \pause
    \end{itemize}
    \item Change of:
    \pause
    \begin{itemize}
      \item Guards
      \pause
      \item Scope
      \pause
      \item Priorities
      \pause
    \end{itemize}
    \item Leads to an expensive refactoring
    \pause
    \item Or expensive rebuilding
    \pause
    \item A concrete example is an IDE
  \end{itemize}
\end{frame}

\begin{frame}
  \frametitle{The Solution To The Problem}
  \begin{quote}
    "It works on my machine." \textemdash Intern
  \end{quote}
  \begin{itemize}
    \item The text editor, the compiler, and the terminal,
    \pause
    \item Eventually needed build scripts for large projects
    \pause
    \begin{itemize}
      \item Lead to more complex build scripts
      \pause
      \item Eventually ending up with applications like Gradle and Maven
      \pause
    \end{itemize}
    \item Missing/incomplete:
    \pause
    \begin{itemize}
      \item Libraries
      \pause
      \item Environment variables
      \pause
      \item Configurations
      \pause
      \item Scripts
      \pause
    \end{itemize}
  \item What if everything is bundled?
  \end{itemize}
\end{frame}

\begin{frame}
  \frametitle{Integrated Development Environment}
  \begin{itemize}
    \item Easier to onboard new developers
    \pause
    \item Other quality of life improvements
    \pause
    \begin{itemize}
      \item File explorer
      \pause
      \item Project manager
      \pause
      \item Version Control System integration
      \pause
      \item Syntax Highlighting
      \pause
      \item Integrated debugging
      \pause
      \item \dots
    \end{itemize}
  \end{itemize}
\end{frame}

\section{Who cares?}
\SectionPage

\begin{frame}
  \frametitle{Bergen Language Design Laboratory}
  \begin{itemize}
    \item Experimenting with a research programming language called Magnolia
    \pause
    \item Takes inspiration from
    \pause
    \begin{itemize}
      \item Generic programming
      \pause
      \item Algebraic specifications
      \pause
      \item Theory of institutions
      \pause
      \item And other languages like CafeOBJ and Maude
      \pause
    \end{itemize}
    \item Created to experiment with novel language features
    \begin{itemize}
      \item Functionalization
      \pause
      \item Mutification
      \pause
      \item Generated types
      \pause
      \item Type partitions
      \pause
      \item \dots
    \end{itemize}
  \end{itemize}
\end{frame}

\begin{frame}
  \frametitle{Magnolia}
  \begin{itemize}
    \item Introduces something called \textit{concepts}
    \pause
    \item Similar to an Java interface.
    \pause
    \item A concept declares
    \begin{itemize}
      \item Types
      \pause
      \item Operations on those types
      \pause
      \item Axioms that specify the behavior of the operations
      \pause
    \end{itemize}
    \item A concept can use other concepts, and rename the types and operations
      in the concept, this is called renaming
  \end{itemize}
\end{frame}

\begin{frame}
  \frametitle{Magma example in mathematical notation}
  \begin{itemize}
    \item Magma
    \begin{itemize}
      \item Set $M$
      \item Binary operation $\bullet$
      \begin{equation}
        \forall a, \forall b \in M \implies a \bullet b \in M
      \end{equation}
    \end{itemize}
  \end{itemize}
\end{frame}

\begin{frame}
    \frametitle{Magma example in Java}
    \begin{center}
      \lstinputlisting
      [ language=Java
      ]{./code/magma.java}
    \end{center}
\end{frame}

\begin{frame}
    \frametitle{Magma example in Magnolia}
    \begin{center}
      \lstinputlisting
      [ language=Magnolia
      ]{./code/magma.mg}
    \end{center}
\end{frame}

\begin{frame}
  \frametitle{Semigroup example in mathematical notation}
  \begin{itemize}
    \item Semigroup
    \begin{itemize}
      \item Set $M$
      \item Binary operation $\bullet$
      \begin{equation}
       \forall a, \forall b \in M \implies a \bullet b \in M
      \end{equation}
      \begin{equation}
        \forall a, \forall b, \forall c \in M \implies
        (a \bullet b) \bullet c = a \bullet (b \bullet c)
      \end{equation}
    \end{itemize}
  \end{itemize}
\end{frame}

\begin{frame}
    \frametitle{Semigroup example in Java}
    \begin{center}
      \lstinputlisting
      [ language=Java
      ]{./code/semigroup.java}
    \end{center}
\end{frame}

\begin{frame}
    \frametitle{Semigroup example in Magnolia}
    \begin{center}
      \lstinputlisting
      [ language=Magnolia
      ]{./code/semigroup.mg}
    \end{center}
\end{frame}

\begin{frame}
  \frametitle{Monoid example in mathematical notation}
  \begin{itemize}
    \item Monoid
    \begin{itemize}
      \item Set $M$
      \item Binary operation $\bullet$
      \begin{equation}
       \forall a, \forall b \in M \implies a \bullet b \in M
      \end{equation}
      \begin{equation}
        \forall a, \forall b, \forall c \in M \implies
        (a \bullet b) \bullet c = a \bullet (b \bullet c)
      \end{equation}
      \begin{equation}
        \forall a, \exists e \in M \implies a \bullet c = a
      \end{equation}
    \end{itemize}
  \end{itemize}
\end{frame}

\begin{frame}
    \frametitle{Monoid example in Java}
    \begin{center}
      \lstinputlisting
      [ language=Java
      ]{./code/monoid.java}
    \end{center}
\end{frame}

\begin{frame}
    \frametitle{Monoid example in Magnolia}
    \begin{center}
      \lstinputlisting
      [ language=Magnolia
      ]{./code/monoid.mg}
    \end{center}
\end{frame}

\begin{frame}
  \frametitle{Group example in mathematical notation}
  \begin{itemize}
    \item Group
    \begin{itemize}
      \item Set $M$
      \item Binary operation $\bullet$
      \begin{equation}
       \forall a, \forall b \in M \implies a \bullet b \in M
      \end{equation}
      \begin{equation}
        \forall a, \forall b, \forall c \in M \implies
        (a \bullet b) \bullet c = a \bullet (b \bullet c)
      \end{equation}
      \begin{equation}
        \forall a, \exists e \in M \implies a \bullet e = a
      \end{equation}
      \begin{equation}
        \forall a, \exists b \in M \implies a \bullet b = e
      \end{equation}
    \end{itemize}
  \end{itemize}
\end{frame}

\begin{frame}
    \frametitle{Group example in Java}
    \begin{center}
      \lstinputlisting
      [ language=Java
      ]{./code/group.java}
    \end{center}
\end{frame}

\begin{frame}
    \frametitle{Group example in Magnolia}
    \begin{center}
      \lstinputlisting
      [ language=Magnolia
      ]{./code/group.mg}
    \end{center}
\end{frame}

\begin{frame}
  \frametitle{If It Ain't Broke}
  The current Magnolia IDE
  \pause
  \begin{itemize}
    \item Integrated with the Magnolia Compiler
    \pause
    \item Made using an old version of Eclipse
    \pause
    \item Uses deprecated Eclipse plugins
    \pause
    \item Installation process is complex
    \pause
    \item In INF220, two weeks is set aside for students to install it
  \end{itemize}
\end{frame}

\section{Why create a \textit{new} IDE?}
\SectionPage

\begin{frame}
  \frametitle{Forking VS Code And Adding AI}
  \begin{itemize}
    \item Current IDEs cannot have good support for all experimental programming
      languages
      \pause
      \begin{itemize}
        \item So niche solutions are needed
          \pause
        \item Which might depend on very specific functionality from the host
          IDE
          \pause
      \end{itemize}
    \item The host IDE could deprecate needed functionality
      \pause
    \item The installation process would then be complex
      \pause
    \item Deep understanding of the host IDE is needed
  \end{itemize}
\end{frame}

\begin{frame}
  \frametitle{Why Modular?}
  \begin{itemize}
    \item Magnolia is still in development
      \pause
    \item The Magnolia toolchain is being developed in parallel
      \pause
    \item Modularity allows for future discoveries to be quickly adopted into
      the IDE
      \pause
    \item Lowers the onboarding time for future maintainers
  \end{itemize}
\end{frame}
